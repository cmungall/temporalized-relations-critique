\documentclass{bioinfo}
%\documentclass{article}
\input{logicmacros}
\usepackage{url}


\ifx\pdfoutput\undefined
% we are running LaTeX, not pdflatex
\usepackage{graphicx}
\else
% we are running pdflatex, so convert .eps files to .pdf
%\usepackage[pdftex]{graphicx}
%\usepackage{epstopdf}
\fi 

\copyrightyear{}
\pubyear{}

\def\partOf{\pr{part\_of}}
\def\hasPart{\pr{has\_part}}
\def\isA{\pr{is\_a}}
\def\instanceOf{\pr{instance\_of}}
\def\derivesFrom{\pr{derives\_from}}
\def\adjacentTo{\pr{adjacent\_to}}

\def\existsAt{\pr{exists\_at}}

\def\partOfAtSomeTimes{\pr{part-of-at-some-times}}
\def\partOfAtAllTimes{\pr{part-of-at-all-times}}
\def\hasPartAtSomeTimes{\pr{has-part-at-some-times}}
\def\hasPartAtAllTimes{\pr{has-part-at-all-times}}
\def\hasPartAtAllTimesForWhichPartExists{\pr{has-part-} \pr{at-all-times-}\ \pr{for-which-part-exists}}
\def\partOfAtAllTimesForWhichWholeExists{\pr{part-of-} \pr{at-all-times-}\ \pr{for-which-whole-exists}}

\def\atAllTimes{\pr{at-all-times}}
\def\atSomeTimes{\pr{at-some-times}}
\def\atAllTimesForWhichSubjectExists{\pr{at-all-times-for-which-subject-exists}}

\def\CellNucleus{\pr{cell nucleus}}
\def\Cell{\pr{cell}}

\def\OBOREL{\textbf{OBO-REL}}

\newcommand{\tbleqn}[1]{
\begin{math}
\begin{aligned}[1]
#1
\end{aligned}
\end{math}
}

\begin{document}
\firstpage{1}

\title{A critique of rigid temporalized relations}

\author{Christopher J. Mungall\,$^{1}$\footnote{to whom correspondence should be addressed}}
\address{$^{1}$Genomics Division, Lawrence Berkeley National Laboratory, MS84R017, 1 Cyclotron Road, Berkeley, CA 94720 USA}

\history{}

\editor{}

\maketitle

\begin{abstract}

  In this review I evaluate the proposed new temporalized relations
  strategy in which many existing relations would be replaced by two
  or more relations, an \emph{at-all-times}\ form and an
  \emph{at-some-times}\ form.

  My findings are that the \emph{at-all-times}\ relations have an
  underlying logical problem that renders them formally incorrect for
  use in many ontologies. The \emph{at-some-times} relations are
  safer, but would lose crucial transitive inferences. These logical
  problems are compounded by the fact that the relations are difficult
  for users and ontology developers to understand, and will most
  likely lead to confusion and errors, especially in the absence of
  detailed documentation.

  I conclude that these relations should not be adopted by ontology
  developers as a replacement for existing relations. Migrating to
  these relations would be an expensive, error-prone process that
  would alienate the user base of an ontology, and the end result
  would be ontologies that are either formally incorrect or too weak
  to perform required inferences.

\end{abstract}

\section{Introduction}

\subsection{The OBO Relations ontology}


The OBO relations ontology (\OBOREL) defined a set of core relations
for use in biological ontologies, including \isA, \partOf\ and
\derivesFrom\cite{Smith2005}. The original \OBOREL\ paper has been
cited 709 times\footnote{Google scholar}, and has been a crucial
reference in the correct usage of relations in biological ontologies.

One notable aspect of \OBOREL\ was the precise specifications of how
relationships change or remain the same through the passage of
time. For example, the intent was that an ontology could state that
\emph{every} \CellNucleus\ is part of \emph{some} \Cell\ at \emph{any}
given moment of time (for which that cell nucleus exists). This has a
precise interpretation in first-order logic:

$
\A x \A t : \instanceOf(x, \CellNucleus, t) \imp \\
 \E y \instanceOf(y, \Cell, t), \partOf(x,y,t)
$

A relationship of this form is known as \emph{permanent generic
  parthood}. It allows for nuclei to be passed from one cell to
another, so long as the nucleus remains part of some cell.

The other notable aspect of \OBOREL\ was the distinction between type
(class) level and instance level relations. Each type-level relation
connects a pair of classes and typically is defined according to an
\pr{ALL-SOME-ALLTIMES}\ pattern - for example, every cell nucleus is
part of some cell at all times. This causes some confusion when using
OWL, which does not support these kinds of class-level
relationships. Instead the relationship between a nucleus and a cell
in OWL is explicitly quantified, but without a time argument, as all
relations in OWL are binary: For example:

\begin{verbatim}
'cell nucleus' SubClassOf part_of some cell
\end{verbatim}

As of 2010, the official semantics of OBO format
have been as a subset of OWL, so what applies to OWL necessarily
applies to OBO.

This impedance mismatch between the OWL interpretation (which can be
considered ``atemporal'') and the \OBOREL\ account has been
problematic in providing a consistent formal account of relations that
is consistent with OWL semantics, although it is not clear that this
has caused a problem for many ontology developers or users. The
standard approach has been to use a set of binary relations as
specified in Table \ref{tab:characteristics-atemporal}, and to assume
an \OBOREL\ type interpretation for time.

\subsection{BFO2 and temporalized relations}

One of the stated goals of the group developing the OWL translation of
version 2 of the Basic Formal Ontology (BFO)\cite{Grenon2004} was to
have relations in OWL handled so as to have a clear First-Order Logic
(FOL) reading according to the BFO reference\cite{BFO2Ref}. The BFO
group explored different strategies for unifying OWL binary properties
with the ternary relations\cite{Grewe}.  One such strategy is the
\emph{temporalized relations} (TR) strategy, in which each reference
relation relating continuants has two or more OWL cognates,
\pr{rel-at-some-times} and \pr{rel-at-all-times}. This strategy has
been adopted as the official one for the OWL translations of
BFO2\cite{Graz}.

\subsection{Outline}

In this review I do not attempt to compare or even describe the
different modeling possibilities (see \cite{Grewe} for details here);
due to pressure to make temporalized relation the standard for OBO
library ontologies I focus purely on this approach.

I first provide an outline of temporalized relations, drawing on the
existing release notes and documentation, attempting to fill some gaps
and provide additional explanations of some of the complexities. I
then present the major problems posed by these relations:

\begin{enumerate}

\item Temporalized relations fail to capture the biological reality,
  forcing ontology editors to make a choice between two unsatisfactory
  models. This would be a giant step backwards in biological
  ontologies.

\item Within the model there is no facility for modeling important classes that lack
  the 'rigidity' criterion.

\item Temporalized relations are complex and confusing for both users
  and experienced ontology editors.

\end{enumerate}

These problems are inter-related; even experienced ontology editors
may not understand the choices they are being asked to make when
migrating to temporalized relations.

Finally I end with some specific recommendations regarding the
temporalized relation strategy in general and BFO2 specifically.

My intentions are to make this review accessible to a wide audience,
and to keep logical formulas to a minimum. This is difficult because
one of the problems with the TR strategy is that it forces complexity
upon both the users and developers of an ontology, necessitating some
discussion of that complexity in order to explain why it does not
yield the intended benefits. I have annexed some of the finer grained
details into an appendix.

\begin{table}
\begin{tabular}{ | p{2.5cm} | p{1cm} | p{1cm} | p{2.5cm} | }
\hline
\textbf{Relation} & \textbf{Trans} & \textbf{Symm} & \textbf{Inverse Of}  \\
\hline
\partOf\  & Yes & No & \hasPart  \\
\hline
\hasPart\ & Yes & No & \partOf \\
\hline
\adjacentTo\ & No & Yes &  \\
\hline
\end{tabular}
\caption{Relation characteristics, atemporal. These are the characteristics of the instance level relations in the current RO (http://obo-relations.googlecode.com)}
\label{tab:characteristics-atemporal}
\end{table}


\section{Temporalized Relations}

Here I distinguish between reference relations (RRs) and their
manifestation in OWL as binary temporalized relations (TRs), using the
temporalized relation strategy (TRS). All relations are instance
level. As a typographic convention I use dashes to separate the words
in a temporalized relation, and underscores in a reference relation.

\subsection{Translation template}

The BFO2 Graz version release notes\cite{Graz} specify a general
template for relating RRs to TRs:

\begin{verbatim}
x rel-at-some-time y ->
 exists(t) exists_at(x,t) -> 
    exists_at(y,t) and rel(x,y,t)
x rel-at-all-times y ->
  forall(t)  exists_at(x,t) -> 
      exists_at(y,t) and rel(x,y,t)
\end{verbatim}

Here I focus on \partOf\ as an exemplar relation, whilst recognizing
that similar patterns may apply to other, but not all relations.

For \partOf\ connecting two continuants there are in fact \emph{three}
TRs rather than two (for reasons that will be explained shortly):

\begin{enumerate}

\item \partOfAtSomeTimes

\item \partOfAtAllTimes

\item \partOfAtAllTimesForWhichWholeExists

\end{enumerate}

Note that in BFO2 the actual labels are ``part of continuant at some
time'', ``part of continuant at all times'' and ``part of continuant
at all times for which whole exists'', I shorten this for brevity --
here we are only concerned with relations that involve a continuant.

\subsection{TRs have a different meaning from OBOREL relations}

On the surface, the \atAllTimes\ form appears to be the same as the
\OBOREL\ interpretation. For example, an ontology that uses TRs may
include the following OWL axiom:

$$
\CellNucleus\ \pr{SubClassOf}\ \partOfAtAllTimes\ \pr{some}\ \Cell
$$

This may seem to be the same as the statement ``every cell nucleus is
part of some cell at all times''. We might even be able to
automatically translate an ontology written using the \OBOREL\
interpretation into TRs. \emph{However, these are NOT the same, and an
  understanding of why this is so is crucial if these relations are to
  be used correctly, and to understand the long-term consequences of
  using them}.

It is not the case that TRs are the same as what has come before, but
with longer labels. \emph{The semantics are fundamentally different},
in ways that may qhave major downstream effects on relation properties and
how relations are used in ontologies.

The most fundamental difference is that \OBOREL\ has a permanent
generic parthood (in which a nucleus must always be part of a cell,
but can be transferred between cells), with existing TRs that
possibility is disallowed. See Table \ref{tab:fol-class-axioms} in the
appendix for details. I will return to this in the evaluation.

The fact that these are different is of utmost importance to how
ontologies are created, and affects the \emph{characteristics} of
these relations in some ways that might seem surprising.

\subsection{Object property characteristics}

In OWL, relations (object properties) can have certain characteristics
such as being \emph{transitive}, \emph{symmetrical}, and they may be
related to other relations via logical axioms such as \emph{inverse
  properties} and property chains. These are extremely useful for many
purposes - transitivity has been at the core of bioinformatics
applications of ontologies from the initial version of the Gene
Ontology\cite{Ashburner2000}, if not before. Inverse properties are
useful for instance level reasoning, and for finding errors in complex
ontologies.

When translating a RR to a TR, it may not be immediately clear what
properties of the RR should be carried over to the TR. If a reference
relation $R$ has an inverse $R'$, it doesn't follow that the
temporalized versions will be inverses. Ideally we would be able to
prove that the RR and TR are consistent, although the proof may be
obvious to a logician. Here I use \partOf\ as an illustrative example,
although each relation may require individual examination for its
properties.

Table \ref{tab:characteristics-temporalized} shows some of the
property characteristics of the various forms of the continuant
parthood relations. Figure \ref{fig:part} shows this in graphical
form.

%----------------------------------------
\begin{figure}
\center
\includegraphics[width=7cm]{part}
\caption{Parthood relations, both atemporal (A) and temporalized
  (B). The atemporal form can be found in the current RO. The
  temporalized form can be seen in the BFO2 Graz release}.
\label{fig:part}
\end{figure}
%----------------------------------------


\begin{table}
\begin{tabular}{ | p{3cm} | p{1cm} | p{4cm} | }
\hline
\textbf{Relation} & \textbf{Trans} & \textbf{Inverse Of}  \\
\hline
\partOf\ \atSomeTimes & No & \hasPart\ \atSomeTimes \\
\hline
\partOf\ \atAllTimes & Yes & \hasPartAtAllTimesForWhichPartExists \\
\hline
\partOfAtAllTimesForWhichWholeExists & Yes & \hasPart\ \atAllTimes \\
\hline
\hasPart\ \atSomeTimes & No & \partOf\ \atSomeTimes \\
\hline
\hasPart\ \atAllTimes & Yes & \partOfAtAllTimesForWhichWholeExists \\
\hline
\hasPartAtAllTimesForWhichPartExists & Yes & \partOf\ \atAllTimes \\
\hline
\end{tabular}
\caption{Relation characteristics for core continuant parthood relations. These characteristics are declared in the current BFO2 OWL Graz version}
\label{tab:characteristics-temporalized}
\end{table}

\subsubsection{Transitivity}

In the case of the RR \partOf\ (which is transitive), the stronger
\atAllTimes\ TR retains the transitivity characteristic, whereas the
weaker \atSomeTimes\ TR does not have this. This means the weaker
version is often (but not always) safer to use in an ontology, but
will in general lead to fewer inferences.

\subsubsection{Symmetricality}

For other relations and other characteristics, the translation may not
be obvious. For example, the adjacency relation is symmetric in its
atemporal form\footnote{I am only considering instance level
  relations}. Should this symmetricality characteristic be carried
over to the temporalized form?

Currently BFO2 does not have an adjacency relation, or any other
reference relation that is symmetric, so the following is based on my
own understanding, and is shown in \ref{tab:characteristics-adj}. I
assume that the symmetricality should be declared for the
\atSomeTimes\ form and not the \atAllTimes\ form[proof to be added
later]. In contrast to the \partOf\ and \emph{transitivity}, here it
is the \emph{weaker} form of relation that inherits the
characteristic.

\subsubsection{Sub Properties}

The general template for a TR hierarchy is one in which \atSomeTimes\
is the most general, with sibling subproperties \atAllTimes\ and
\atAllTimesForWhichSubjectExists. This is illustrated for parthood in
figure \ref{fig:part}.

[need more verbiage on why the hierarchy is like this. proofs?]

Note how this interacts with other properties. If:
$$x\ \adjacentTo\ \atAllTimes\ y$$

then we can infer that 

$$y\ \adjacentTo\ \atSomeTimes\ x$$

\subsubsection{Inverse Properties}

Ontologies frequently declare inverse relations. For example, the
RR \partOf\ is the inverse of \hasPart. This is fairly standard
practice, and the inverse relations are extremely useful for
reasoning. In OWL, it's not strictly necessary to declare an inverse,
as it is possible to use an InversePropertyExpression. Here I take
the view that inverses that have typically been declared in previous
ontologies are useful, and should also be declared as RRs and have
corresponding TRs.

The inverse of the TRs of \partOf\ may not be completely
intuitive. Naively we might guess that \partOfAtAllTimes\ would be the
inverse of \hasPartAtAllTimes, but this not the case.

In fact, the declaring the inverse of \partOfAtAllTimes\ requires
declaring a third, more specific, TR form,
\hasPartAtAllTimesForWhichPartExists. The
\atAllTimesForWhichSubjectExists\ form can be generated for some but
not all RRs. Whilst it complicates the ontology of TRs to introduce
this extra form, the alternative of not having inverses (or of being
forced to write complicated inverse expressions) may be too
prohibitive for many users.

\subsubsection{Other characteristics}

I do not consider other characteristics such as anti-symmetry and
domain/range restrictions here[perhaps future versions].

\subsubsection{Generating a TR from an RR}

Currently there is no ``recipe'' for generating a set of TRs from an
RR. Different patterns may apply to different RRs.

For example, when creating the TRs for \pr{inheres\_n} and its inverse
\pr{bearer\_of}, it is correct to declare \pr{bearer-of-at-some-times}
InverseOf \pr{inheres-in-at-some-times}. This is because the inherence
relation is non-migratory[link to tracker].

\begin{table}
\begin{tabular}{ | p{3cm} | p{1cm} | p{4cm} | }
\hline
\textbf{Relation} & \textbf{Symm} & \textbf{Inverse Of}  \\
\hline
\adjacentTo\ \atSomeTimes & Yes & \adjacentTo\ \atSomeTimes \\
\hline
\adjacentTo\ \atAllTimes & No & \\
\hline
\end{tabular}
\caption{Relation characteristics for a typical symmetric relation. This relation is not part of the BFO2 OWL Graz version}
\label{tab:characteristics-adj}
\end{table}

\section{Evaluation}

My evaluation is split into two parts: (1) the unsuitability of the
underlying TR logical formalism for ontologies that deal with change
and (2) the inherent complexity of the TR relations.

\subsection{Temporalized Relations misrepresent the biology}

TRs present many challenging problems. For example, when converting an
anatomy ontology that has been modeled traditionally using the
relations in table \ref{tab:characteristics-atemporal}, the ontology
editor must make a choice on a case by case basis as to which of
the relations in \ref{tab:characteristics-temporalized} should be
used.

This is an onerous task, but this could be justified if the results
were better ontologies. However, in many cases \emph{none of the
  choices are appropriate}, resulting in an ontology that is
\emph{worse} in terms of formal correctness and reasoning power.

\subsubsection{Ontologies require permanent-generic relationships}

One reason why TRs are formally incorrect for many ontologies is
because most appropriate choice of parthood relationship is the
\emph{permanent-generic} form, as exemplified in the original OBO
relations paper. The standard example here is the relationship between
a cell nucleus and a cell. At any moment in time, a given cell nucleus
is by definition part of some cell\footnote{we would consider extruded
  nuclei to be transformations of cell nuclei, but instantiating a
  different class}. However, \emph{this need not be the same cell
  throughout the lifetime of the nucleus}.

In contrast, if the \partOfAtAllTimes\ TR is used then the
interpretation is that the cell nucleus is always part of the same
cell. This interpretation can be proved to be formally wrong in cells
that undergo cell division\cite{CellDiv}. This is because in reality
the cell nucleus is always part of a cell, \emph{but not the same
  cell}. With \partOfAtAllTimes\, there is no ``migration'' allowed -
a nucleus is always part of the same cell.

Understanding the consequences of incorrectly modeling the
relationship in this way are not straightforward. In some situations
there may be no immediate problem if the ontology is not used for
instance data. However, even in these circumstances, it would be
unusual to adopt a far more complex formalism (TRs) in the name of
formality only to arrive at an ontology that is formally incorrect.

Given that the ontology editor can not use \partOfAtAllTimes\ without
making a false statement, they must choose the weaker qualified
version of \partOf\ such as \partOfAtSomeTimes. Such a usage would be
formally \emph{valid}, but \emph{incomplete} from the point of view of
useful reasoning. This is because the weaker \atSomeTimes\ form lacks
the transitivity characteristic. I assume no disagreement that this is
unsatisfactory; most ontologies are dependent on parthood
transitivity.

A third possibility is to use
the \partOfAtAllTimesForWhichWholeExists\ form, but this would also be
wrong in this case[need more info here - todo - proofs].

In the nucleus-cell case, the ontology editor requires the permanent
generic form in order to be both accurate and to get the required
inferences. However, this form is specifically excluded in the existing
TR strategy.

\subsubsection{Temporalized relations do not work for structures that
  change}

There are many cases beyond the nucleus-cell case where permanent
generic parthood is required. In fact, the logical problem with the
TRs is deeper than the permanent-generic relationships, and extends to
a much wider range of ontology classes, including classes whose
members change with respect to some property (such as location) over
time.

As an example, consider the class 'premigratory neural crest
cell'. This is a cell that is part of the neural crest region of the
neuroepithelium, prior to migration. This might conventionally be
modeled in OWL as being a SubClass of \partOf\ some \pr{neural
  crest}. In choosing the correct TR for this relationship we have the
choice of whether to go with the weaker \atSomeTimes\ or the stronger
\atAllTimes. Again, the weaker relation loses transitivity; the
stronger relation is formally wrong, because the cells can migrate. It
is formally incorrect to say they are part of a neural crest at all
times for which they exist - the correct statement is to say that they
are part of the neural crest at all times \emph{for which they
  instantiate the premigratory neural crest cell class}. This is the
statement that is allowed with \OBOREL\ semantics but explicitly not
an option with the TR approach. This brings us to the next, related,
restriction, the inability of the TR approach to represent
\emph{non-rigid} classes such as 'premigratory neural crest cell' (see
next section).

To summarise the criticisms so far: the correct relationship to use in
many cases where change is involved is the \emph{permanent-generic},
yet this option is out of scope with existing TRs, forcing ontology
developers to make a choice between sub-optimal relations. The TR
approach also forbids us from accurately representing many
relationships that change over time.

\subsection{Rigidity requirement is too onerous}

The TR approach disallows instantiation of classes that are
\emph{rigid}, which is a severe constraint when developing ontologies
that deal with things that change over time.

The BFO2 Graz release notes\cite{Graz} state:

\begin{quotation}

  Thus we only instantiate ``rigid'' classes, as the interpretation we
  take is a rdf:type C $\Rightarrow$ forall(t) a exists at t $\imp$ a
  instance of C at t. Temporally restricted instantiation is not
  supported in this version of BFO in OWL. We are working on it for
  the future.

\end{quotation}

To many users this may seem like an obscure point, but this is
actually a severe restriction. A class is rigid if it is instantiated
``for life''. If an individual transforms from being an instance of
one class of thing to another class, then those classes are not rigid.

An example of a rigid class may be ``Homo sapiens''. If an individual
instantiates this class at some time t, then they instantiate it all
times for which they exist (barring some unusual inter-species
transformation). Upper level categories like 'process' are also rigid.

The following classes are non-rigid, and therefore do not have full
support in this version of BFO / the TR approach:

\begin{enumerate}

  \item 'premigratory neural crest cell'
  \item 'human with Parkinson's disease'
  \item 'female organism'
  \item 'infected lung'
  \item 'professor' (but \emph{professor role} is allowed)
  \item 'human patient' (but \emph{patient role} is allowed)
  \item 'oocyte'
  \item 'fractured bone'
  \item 'happy human'
  \item 'fetal heart'
  \item 'open heart valve'
  \item 'gravid uterus'
  \item 'phosphorylated EGFR protein'
  \item 'cytoplasmic NFkB'
  \item any leaf node from PATO

\end{enumerate}

See the appendix for a full discussion of each of these cases.

In some cases the constraint may not be so onerous. It can be argued
that a well-structured ontology would never include a class
``professor'', and that this should always be modeled using a rigid
class (human) plus a role (professor role). However, these decisions
should be made on a case by case basis for each ontology that than
imposed from above.

In many of other other cases, forcing the ontology developer to
exclude some of the classes above is too onerous. For example, many
anatomical ontologies make use of phase or stage as a differentium.

When considering BFO2 in particular, there is an interesting
disjunction between the reference document, which explicitly states
that \emph{determinates} (for example, qualities such as ``square'',
``charged'', ``cylindrical'') are non-rigid. The Graz release states
that instantiation of these classes is not supported. These two
seemingly contradictory statements are not explicitly linked
anywhere. The modeling implications of this disconnect are not clear,
and require further documentation. It cannot be ruled out that this
restriction will involve further complexity.

Whilst technically the BFO2 release notes only state that
instantiation of non-rigid classes is not supported, I showed
previously with the neural crest cell example that non-rigid classes
are not compatible with the TR approach. It is impossible to
adequately represent the parthood relationships for classes such as
'premigratory neural crest cell' using the existing TR strategy.

It may be the case that future versions of the TR strategy will allow
for non-rigid classes. It is not clear how this will be achieved
without additional complexity.  The TR strategy must be evaluated on
what exists presently, rather than what it might become in the future,
and at this time the strategy comes with major constraints that
ontology developers should be fully aware of.

To summarize this criticism: non-rigid classes are commonly used in
many biological ontologies. Instantiation of these classes are not
supported in this version of the TR approach, and relationships cannot
be effectively specified for these classes using the TR approach.

\subsection{Temporalized relations add considerable complexity}

\subsubsection{Temporalized relations proliferate relations in a complex network}

The most striking feature of an ontology that uses the TR strategy is
the complexity. Whereas using traditional modeling, we may have has a
single parthood relation, we now have three, arranged in a
counter-intuitive network of axioms; constrast (A) and (B) in figure
\ref{fig:part}. Note that in fact this figure is a simplification, as
it does not show other sub-properties, such as the member part
relations. In production ontologies, most relations involving
continuants would have to be split in this way. Many relations are
inter-related via property chains and other axioms, it is as yet
unclear how much complexity an ontology rich in relations would
suffer.

\subsubsection{Multiple levels of quantification}

Another source of complexity is that ontology editors now have to
handle an extra layer of quantification. Consider some of the possible
ways to model the relationship between a population of organisms and
an organism:

\begin{verbatim}
1. population SubClassOf 
  has-part-at-some-time some organism
2. population SubClassOf 
  has-part-at-all-times some organism
3. population SubClassOf 
  has-part-at-all-times-that-part-exists
    some organism
4. population SubClassOf 
  has-part-at-some-time only organism
\end{verbatim}

In each case there is in fact three levels of quantification. The
first level is the OWL subclass axiom, which states that the condition
holds for ALL instances of a population. Also within the scope of OWL
is the final SOME or ONLY quantifier. Finally, embedded within the
relation is an additional layer is the temporal quantification. Note
that this final layer is opaque to OWL reasoners (and thus harder to
use standard tools to check).

This is in contrast to the simpler, well-documented kind of atemporal
quantification ontology developers perform at the moment.

(Note that \emph{none} of the axioms above are correct, given that
populations gain and lose members over time - permanent generic
parthood is required).

I have a great deal of experience in training and assisting ontology
developers in the use of tools such as reasoners and in making the
transition to OWL. In my estimation, the level of complexity TRs exert
is simply too high.

\subsubsection{Migration is complex and will be error-prone}

In theory some of this complexity could be tamed by additional tooling
(although it is not clear who has the resources to implement
this). However, even if this complexity can be hidden from the user,
the ontology developer is forced to wrestle with the complexity if
they are to use the relations correctly.

This complexity first manifests when an ontology developer chooses to
migrate from a traditionally modeled ontology using relations from
table \ref{tab:characteristics-atemporal}, assuming an \OBOREL\
interpretation.

For each axiom that uses a relation that has multiple variants in TR
form, the developer must make a choice of which of the 3 variants to
use. Surprisingly, there is no documentation or guidance on how they
should do this (meaning mistakes will be made).

A conservative strategy would be to convert all continuant relations
to the \atSomeTimes\ form. This would result in the ontology being far
less useful for inference (due to the loss of properties such as
transitivity). However, we might expect it to be at least valid, since
the relation is weaker.

In fact converting to the \atSomeTimes\ form is not universally
safe. Consider the OWL axiom:

\begin{verbatim}
(part_of some nucleus) DisjointWith
   (part_of some cytoplasm)
\end{verbatim}

These kinds of spatial disconnectedness axioms are very useful for
error detection.

Converting these to \atSomeTimes\ would actually result in an axiom
that is \emph{too strong}. Conversely, converting to \atAllTimes\
would be too \emph{weak}, because it would admit the possibility of
migratory structures being part of two spatially disconnected
locations at the same time (so long as they weren't permanently part
of each). It is not clear how the ontology maintainer should convert
this axiom -- \emph{because all choices are suboptimal}.

Even in cases where there is an optimal way to translate to TRs,
performing the conversion requires an ontology developer who has a
strong understanding of the domain and of the logic.

\subsubsection{Summary of complexity criticism}

To summarise this criticism: TRs introduce considerable complexity,
and without both training and tool support ontology developers and
users are likely to use them incorrectly.



\subsection{Case study: HDOT}

...

\section{Discussion}


\subsection{Keep it as simple as possible, but not simpler}

Many engineers abhor solutions that appear overly baroque or complex
compared to the task at hand. From the point of view of the person
proposing the complex solution, this can be frustrating and seem like
short-sightedness on the part of the engineers. Perhaps they lack the
experience or vision to foresee the time in the future when the
complexity of the solution will be justified.

Nevertheless, I believe we should take heed of engineers'
concerns. Extraordinary complexity demands extrordinary justification.
I have discussed the TR solution with a number of people with
expertise ranging from ontology engineering, bioinformatics, biology
and formal logic. Most of them find the TR solution excessively
complex, unjustified by any use case, and those with an understanding
of the logic realize the underpinnings are flawed. Perhaps we are
collectively shortsighted or missing some aspect of the bigger picture
- I don't think so. But even if this were the case, the fact that such
a large section of the community finds the solution to be so
unworkable demands means that the proponents of the solution need to
work especially hard to justify the complexity.

\subsection{On the strengths and weaknesses of OWL}

One of the stated goals of the BFO2 OWL project is to have a clear FOL
reading of the OWL according to BFO. This is a laudable aim but must
be balanced against the needs of real-world users of ontologies. Their
requirements should not be trumped by the desire for formal perfection.

We must also consider whether OWL is the best mechanism for achieving
this kind of perfection. OWL is by design more restricted than
first-order logic (which is itself arguably inadequate to model
biology in anything other than a simplistic fashion). These
restrictions make it more suited to certain kinds of tasks than
others. In my experience OWL is tremendously useful for building and
maintaining terminological networks that model the world in a very
simplistic but very useful fashion. Perhaps it is the case that
attempting a perfect FOL reading of an OWL version of the BFO2 spec is
simply using OWL in the wrong way?

In fact it may be possible to have an alternate way of modeling the
BFO2 reference in OWL that has more in common with the \OBOREL\
approach and does not ask the users and developers of ontologies to
compromise so much. This is outside the scope of this paper, see
\cite{Grewe} for details.

\subsection{Recommendations}

Based on my evaluation I make the following recommendations:

\subsubsection{Do not use} My primary recommendation is that
Temporalized Relations should not be used as a replacement for
existing atemporal relations. Most ontologies should not migrate.

\subsubsection{Documentation} The TR strategy needs much more
documentation if ontology developers are to use TRs. Even if TRs are
abandoned in their current form (as I recommend), more documentation
would be useful to be able to help achieve consensus on this matter.

\subsubsection{Alternate strategies} Given the inherent limitations
and complexity of TRs, adequate consideration should be given to
alternate strategies such as Temporally Qualified Continuants
(TQCs). The ``default'' strategy of continuing to use simple OWL
object properties as if they has a \OBOREL\ interpretation should be
the default strategy until an adequate replacement is found.

\subsubsection{Use cases} If adopted, TRs will require tremendous
effort in ontology migration, documentation and tooling. There is
little to motivate ontology developers to do this as the existing
default strategy works for them. The main motivating factor seems to
be a desire for formal correctness, at the expense of usability and
biological correctness. 

\subsubsection{Road map} The existing TR proposal embodied in the BFO2
Graz release is not complete. For example, of non-rigid classes, the
release notes state that ``we are working on [non-rigid classes] for
the future''. There shoould be a roadmap indicating when these
solutions are expected to transpire. Ontology developers should not be
expected to commit production ontologies to an experimental project
with no roadmap.

%\subsubsection{Smooth traAnsition} If TRs are to be adopted, there
%needs to be some incremental transition plan for migration.

\section{Conclusions}

Temporalized Relations would be a massive fundamental change to the
way relationships are modeled in ontologies. They would introduce
significant additional complexity to both users and developers of
ontologies.

Some of these costs could be justified if Temporalized Relations were
on a path towards making ontologies more biologically
accurate. However, there are no mativating use cases for this
transition, and in fact migrating to Temporalized Relations would lead
to ontologies becoming \emph{less} accurate, in addition to more
complex.

My recommendation is unambiguous in its rejection of the use of
Temporalized Relations in their current form in biological ontologies.

\section*{Acknowledgments}



%\section*{References}

% ========================================
\bibliography{trc}
\bibliographystyle{plain}
% ========================================

\newpage
\section*{Appendix}

\subsection{Non-rigid classes}

A class $c$ is non-rigid if there exists an instance $i$ that exists
at $t_1$ and exists at $t_2$, and it is not the case that $i$
instantiates $c$ at $t_1$, and $i$i instantiates $c$ at $t_2$. $t_1$
may precede or succeed $t_2$.

\begin{enumerate}

\item 'premigratory neural crest cell' -- This is a cell that is part
  of the neural crest region of the neuroepithelium, prior to
  migration. Whilst not all instances of this class become migratory,
  many do. In this case we have a cell instance $i$ that instantiates
  'premigratory neural crest cell' at $t_1$ and then at a subsequent
  time $t_2$, it instantiates 'migratory neural crest
  cell'. Therefore, 'premigratory neural crest cell' is a non-rigid
  class. The \partOf\ relationship between this class and the neural
  crest cannot be adequately represented using TRs, because the cells
  may migrate.

\item 'human with Parkinson's disease' -- a person is not born with
  Parkinsons (although they may be born with genes that
  predispose). It is possible for a human being $i$ who exists at
  $t_1$ and $t_2$ to not instantiate human-with-PD at $t_1$ and to
  instantiate human-with-PD at $t_2$. Therefore human-with-PD is
  non-rigid. The \pr{has\_disposition}\ relationship between this class
  and the disease class cannot be adequately represented using TRs, as
  we need to say the members of the class have the disposition for all
  times that they instantiate the class.

\item 'female organism' -- Some organisms (e.g. some species of
  arthropod) can change sex during their lifetime. It is possible for
  some such instance $i$ to instantiate 'male organism' at $t_1$ and
  then instantiate 'female organism' at $t_2$. If these classes are
  disjoint, then 'female organism' is non-rigid. Note that it is
  possible to define different sex concepts (gender, karyotypic sex,
  biological sex, ...), different arguments can be made about the
  rigidity of the corresponding material entity classes.  The
  \pr{has\_quality}\ relationship between this class and the sex
  quality class cannot be adequately represented using TRs, as we need
  to say the members of the class have the quality for all times
  that they instantiate the class.


\item 'infected lung' -- it is possible for a lung $i$ to have the
  quality of being infected (alternatively: be the location of a
  population of invading organisms) at $t_1$, and then non-infected at
  $t_2$. Therefore the class 'infected lung' is non-rigid.  The
  \pr{location\_of}\ relationship between this class and the population
  class cannot be adequately represented using TRs, as we need to say
  the members of the class are the location for all times that they
  instantiate the class.


\item 'professor' -- It is possible for an individual $i$ to
  instantiate 'human with professor role' at one time, and then not
  instantiate this at some later time. Therefore 'professor' (as a
  material entity) is non-rigid. professorhood is best represented as
  a role that can be gained or lost.

  \item 'human patient' -- see 'professor'

  \item 'fractured bone' -- It is possible for some bone $i$ to
    instantiate the class 'non-fractured bone' at $t_1$ and then
    'fractured bone' at some later time $t_2$. Therefore 'fractured
    bone' (in the sense of a material entity - a bone that has the
    quality of being fractured) is a non-rigid class. An argument can
    be made that $i$ ceases to exist when it becomes fractured, and is
    replaced by a new individual $i_2$ at $t_2$. This could certainly
    be argued for severe breakages, where $i$ is replaced to two or
    more bone shards.

  \item 'happy human' -- It is possible for a human being $i$i to
    instantiate the class 'happy human' at $t_1$ (by virtue of bearing
    a happy disposition) and then 'unhappy human' at $t_2$. Therefore
    'happy human' is a non-rigid class. The \pr{has\_disposition}\
    relationship between this class and the disposition class cannot
    be adequately represented using TRs, as we need to say the members
    of the class have the disposition for all times that they
    instantiate the class.

  \item 'fetal heart' -- it is possible for a heart $i$ to instantiate
    'fetal heart' at $t_1$ and and then 'newborn heart' at
    $t_2$. Therefore 'fetal heart' is non-rigid. An argument can be
    made that $i$ is ceases to exist and is replaced by a new instance
    $i_2$ at $t_2$, but this would be unusual. An argument could also
    be made that there is no need for a class 'fetal heart' - the
    concept should be described using a rigid class 'heart' together
    with an occurrent 'fetal stage'. However, this would be a severely
    onerous penalty on many anatomy ontologies which frequently use
    stage as a differentia.

  \item 'open heart valve' -- it is possible for some heart valve $i$
    to instantiate 'open heart valve' at $t_1$ (by virtue of bearing
    the quality 'open' at this time, or, alternatively, by virtue of
    their being a lumen in the vessel) and then to instantiate 'closed
    heart valve' at $t_2$. Therefore 'open heart valve' is a non-rigid
    class.

  \item 'gravid uterus' -- it is possible for some uterus $i$ to
    instantiate a class 'non-gravid uterus' at time $t_1$ (by virtue
    of not being the location of a developing organism), and then
    instantiate a class 'gravid uterus' at some later time
    $t_2$. Therefore 'gravid uterus' is a non-rigid class. The
    \pr{has\_part}\ class axiom between this class and the embryo
    class cannot be adequately represented using TRs, as we need to
    say the members of the class are the location for all times that
    they instantiate the class.


  \item 'phosphorylated EGFR protein' -- there are different ways to
    model this depending on identity conditions on the instance level
    (we take identity conditions on the class level as being
    uncontroversial - class equivalence is determined by structure for
    molecules). Using model $M_1$, we assume there to be a single
    instance $i$ of an EGFR protein which transitions through
    different states. Here, $i$ instantiates 'unphosphorylated EGFR'
    at $t_1$ and then later the same instance $i$ instantiated
    'phosphorylated EGFR' at $t_2$. Under this model, 'phosphorylated
    EGFR' is non-rigid. We can model this differently - call this
    $M_2$. Here $i_1$ instantiates 'unphosphorylated EFGR' at
    $t_1$. Then, as a phosphate group is added at $t_2$, $i_1$ ceases
    to exist and its place is taken by $i_2$, which instantiates
    'phosphorylated EGFR'. Here $i_1$ and $i_2$ might be related via
    some relation such as 'transformation of'. This illustrates that
    any non-rigid class can be made rigid by changing instance-level
    identity conditions. At one extreme we can see life as a series of
    snapshots, with individuals living for an instant before being
    replaced by a doppleganger.

  \item 'cytoplasmic NFkB' -- this is similar to the EGFR case. Here
    the differentia is location. No structural change need take
    place. The \pr{located\_in}\ or \pr{part\_of}\ class axiom between
    this class and the cytoplasm class cannot be adequately
    represented using TRs, as we need to say the members of the class
    are part of the cytoplasm for all times that they instantiate the
    class.

  \item any leaf node from PATO -- Examples: square, open,
    cylindrical, hot, cold. This is a multi-faceted topic and a
    thorough discussion should wait until there is documentation on
    how to model quantities in BFO2.

\end{enumerate}

\subsection{First order logic axioms}

This section contains some additional material on the first order
logic axioms supporting the Temporalized Relations. 

Table \ref{tab:fol-class-axioms} shows the FOL for the two main
temporalized versions of \partOf. Table
\ref{tab:fol-part-temporalized} shows the full FOL semantics of making
OWL class axioms using the FOL relations.


\begin{table}
\begin{tabular}{ | p{3cm} | p{4cm} | }
\hline
\textbf{Relation} & \textbf{Axiom}  \\
\hline
$$x\ \pr{part-of-}\ \atAllTimes\ y$$ &
        \tbleqn{
 & \dimp  \A t \existsAt(x,t) \imp\ \\
 & \existsAt(y,t), \partOf(x,y,t)
} \\
\hline
$$x\ \pr{part-of-}\ \atSomeTimes\ y$$ &
        \tbleqn{
 & \dimp  \E t \existsAt(x,t) \imp\ \\
 & \existsAt(y,t), \partOf(x,y,t)
} \\
\hline
\end{tabular}
\caption{Temporalized relations axioms for parthood relations. Taken from \cite{Graz} and transcribed into FOL syntax}
\label{tab:fol-part-temporalized}
\end{table}


\begin{table}
\begin{tabular}{ | p{1.8cm} | p{6.2cm} | }
\hline
\textbf{Axiom}  \\
\hline
\partOf (\OBOREL) &
        \tbleqn{
 & \A x \A t : \instanceOf(x, \CellNucleus, t) \imp \\
 & \E y : \instanceOf(y, \Cell, t), \partOf(x,y,t)
} \\
\hline
\pr{part-of-} \atAllTimes &
        \tbleqn{
 & \A x \E t : \instanceOf(x, \CellNucleus) \imp \\
 & \E y \instanceOf(y, \Cell), \A t \existsAt(x,t) \imp\ \\
 & \existsAt(y,t), \partOf(x,y,t)
} \\
\hline
\pr{part-of-} \atSomeTimes &
        \tbleqn{
 & \A x \E t : \instanceOf(x, \CellNucleus) \imp \\
 & \E y \instanceOf(y, \Cell), \E t \existsAt(x,t) \imp\ \\
 & \existsAt(y,t), \partOf(x,y,t)
} \\
\hline
\end{tabular}
\caption{
  Semantics of class axioms with parthood example. The first row shows the
  biologically correct relationship (permanent generic parthood), given by \OBOREL\ semantics. The next two rows show 
  two of the temporalized options - neither of these is equivalent to the \OBOREL\ version. 
}
\label{tab:fol-class-axioms}
\end{table}


\end{document}
