\documentclass{bioinfo}
%\documentclass{article}
%%%%%%%%%%%%%%%%%%%%%%%%%%%%%%%%%%%%%%%%%%%%%%%%%%%%%%%%%%%%%%%
%%                    ** mymacros.tex **                     %%
%%%%%%%%%%%%%%%%%%%%%%%%%%%%%%%%%%%%%%%%%%%%%%%%%%%%%%%%%%%%%%%


\typeout{** loading mymacros.tex **}

\def\h{\hbox}

\newcommand{\mathcom}[3]{ \newcommand{#1}[#2]{\mbox{$#3$}}}
\newcommand{\remathcom}[3]{ \renewcommand{#1}[#2]{\mbox{$#3$}}}

\newcommand{\mdef}[2]{ \newcommand{#1}{\mbox{$#2$}} }
 
\mathcom{\y}{0}{\ \vdash\ }               % yeilds 
\mathcom{\ys}{0}{\vdash_{\cal S}}         % yeilds sub S
\mathcom{\yt}{0}{\vdash_{\theta}}         % yeilds sub theta

\mathcom{\absurd}{0}{\mathbf{f}}                 % absurdity
\mathcom{\imp}{0}{\ \rightarrow\ }            % implication arrow
\mathcom{\rimp}{0}{\ \leftarrow\ }            % implication arrow

\mathcom{\con}{0}{\ \wedge\ }                 % conjunction
\mathcom{\dis}{0}{\ \vee\ }                   % disjunction
\mathcom{\n}{0}{\neg}                     % negation
\mathcom{\dimp}{0}{\ \leftrightarrow\ }       % mat equiv

\mathcom{\corresponds}{0}{\ \Lleftarrow\! \! \Rrightarrow\ }


%%\mathcom{\th}{0}{\theta}                  % theta


\mathcom{\A}{0}{\forall}                  % universal quantifier
\mathcom{\E}{0}{\exists}                  % existential quant

\mathcom{\Dec}{1}{{\cal D}(#1)}           % D(#1)

\mathcom{\elt}{0}{\in}

\mathcom{\tuple}{1}{\langle #1 \rangle}

% \mathcom{\equivdef}{0}{\equiv_{\mbox{\em \tiny def}}}
% \mathcom{\eqdef}{0}{=_{\mbox{\em \tiny def}}}
\def\equivdef{\mathrel{\ \equiv_{\mbox{\em \tiny
def}}\ }}

\def\eqdef{\mathrel{\ =_{\mbox{\em \tiny def}}\ }}


% \newtheorem{Rule}{Rule}
% \newtheorem{Lemma}{Lemma}
% \newtheorem{Theorem}{Theorem}


% Redeclaration of symbols for intuitionistic logic
% see Diller p.126
\def\ineg{\mathop\sim}
\def\iimp{\mathbin\Rightarrow}
\def\iiff{\mathbin\Leftrightarrow}
% new symbols from amssymb
%\def\iimp{\mathbin\rightarrowtail}
%\def\iiff{\mathbin\leftrightsquigarrow}

\def\icon{\mathbin\curlywedge}
\def\idis{\mathbin\curlyvee}

%Definitions for RCC inserts
\newcommand{\notch}{\Longrightarrow\kern-16pt{/}\ \ \ }
\newcommand{\ch}{\Longrightarrow}
\newcommand{\pr}[1]{\mbox{\sf #1}}

% RCC relations
\mathcom{\C}{0}{\pr{C}}
\remathcom{\P}{0}{\pr{P}}
\remathcom{\O}{0}{\pr{O}}
\mathcom{\PO}{0}{\pr{PO}}
\mathcom{\PP}{0}{\pr{PP}}
\mathcom{\NTPP}{0}{\pr{NTPP}}
\mathcom{\NTPPI}{0}{\pr{NTPP}^{-1}}
\mathcom{\TPP}{0}{\pr{TPP}}
\mathcom{\TPPI}{0}{\pr{TPP}^{-1}}
\mathcom{\TP}{0}{\pr{TP}}
\mathcom{\NTP}{0}{\pr{NTP}}
\mathcom{\NTPi}{0}{\pr{NTPi}}
\mathcom{\EC}{0}{\pr{EC}}
\mathcom{\DC}{0}{\pr{DC}}
\mathcom{\DR}{0}{\pr{DR}}
\mathcom{\EQ}{0}{\pr{EQ}}
\mathcom{\CG}{0}{\pr{CG}}

\mathcom{\Pin}{0}{\pr{Pi}} % \Pi is Greek letter
\mathcom{\PPi}{0}{\pr{PPi}}
\mathcom{\NTPPi}{0}{\pr{NTPPi}}

\mathcom{\OP}{0}{\pr{OP}}

\mathcom{\conv}{0}{\pr{conv}}
\mathcom{\rsum}{0}{\pr{sum}}
\mathcom{\rdiff}{0}{\pr{diff}}
\mathcom{\rcompl}{0}{\pr{compl}}
\mathcom{\rprod}{0}{\pr{prod}}
\mathcom{\cp}{0}{\pr{cp}}
\mathcom{\Us}{0}{\pr{Us}}
\mathcom{\us}{0}{\pr{u}}
\mathcom{\NULL}{0}{\pr{NULL}}
\mathcom{\rnull}{0}{\emptyset}
\mathcom{\CONV}{0}{\pr{CONV}}
\mathcom{\CON}{0}{\pr{CON}}

% Adaptations for new latex compatibility
\renewcommand{\Box}{\square}
\def\cal{\mathcal}

% meta language in definitions etc
\mathcom{\ml}{1}{\;\;\;\; \mbox{#1} \;\;\;\;}


%-- TOPLOG SYMBOLS

\mathcom{\Co}{0}{{\cal C}_{0} \ }
\mathcom{\CoX}{0}{{\cal C}_{0}}
\mathcom{\Cop}{0}{{\cal C}_{0}^+ \ }
\mathcom{\CopX}{0}{{\cal C}_{0}^+}
\mathcom{\mcop}{0}{\models_{C_o^+}}

\mathcom{\Lop}{0}{{\cal L}_{0}^+ \ }
\mathcom{\LopX}{0}{{\cal L}_{0}^+}
\def\mlo{\models_{{\mathcal  L}_0}}
\def\mlop{\models_{{\mathcal  L}_0^+}}

\mathcom{\Ci}{0}{{\cal C}_{1} \ }
\mathcom{\CiX}{0}{{\cal C}_{1}}

\mathcom{\Io}{0}{{\cal I}_{0} \ }
\mathcom{\IoX}{0}{{\cal I}_{0}}
\mathcom{\Iop}{0}{{\cal I}_{0}^+ \ }
\mathcom{\IopX}{0}{{\cal I}_{0}^+}

%%\mathcom{\inter}{1}{\langle #1 \rangle}
\mathcom{\inter}{1}{i( #1 )}

\newfont{\myssfont}{cmss12}

\mathcom{\Maps}{2}{\, _{#1}\!\!\!\rightleftharpoons^{#2} \ }

\mathcom{\CotoST}{0}{\Maps{C_0}{ST}}
\mathcom{\CtoST}{0}{\Maps{C\,}{ST}}
\mathcom{\IotoST}{0}{\Maps{I_0}{ST}}

\mathcom{\uni}{0}{\cal U \, }
\mathcom{\uniX}{0}{\cal U}
\mathcom{\U}{0}{\cal U}

\mathcom{\eu}{0}{\,=\,\uni}

\mathcom{\ol}{1}{\overline{#1}}

\mathcom{\Lsse}{0}{{\cal L}_{sse}}
\mathcom{\Lssei}{0}{{\cal L}_{ssei}}

\mathcom{\Luse}{0}{{\cal L}_{use}}
\mathcom{\Lusei}{0}{{\cal L}_{usei}}

\mathcom{\disdots}{0}{\dis\!\!\!\ldots\!\!\dis}
\mathcom{\condots}{0}{\con\ldots\con}





%% Miscelleneous


\def\HA#1#2{\pr{Holds-At}(#1,#2)}
\def\col{\!:\!}

% Enclose arguments in square double brackets to
% refer to semantic value of an expression
\def\semvalue#1{[\mkern-3mu[#1]\mkern-3mu]}

\def\forces{\mathop{\setbox0=\hbox{$\vdash$}\ \rule{0.04em}{1\ht0}\mkern1.5mu\box0}}

\mathcom{\hence}{0}{\Longrightarrow \hspace{2em}}

\def\sliderule{\centerline{\rule{4in}{1mm}}}

\def\Cbox{\mathop{\square\mkern-10mu
                   \mbox{\raise0.45ex\hbox{\tiny\sf C}}
                  \mkern2mu
                 }
         }

\def\letterbox#1{\mathop{\square\mkern-10mu
                   \mbox{\raise0.2ex\hbox{\small\sf #1}}
                  \mkern2mu
                 }
         }

%\def\tBox{\letterbox{\lower0.4ex\hbox{t}}}
\def\sBox{\letterbox{s}}
\def\tBox{\letterbox{t}}

\def\overstrike#1#2{\mathop{%
               \setbox1=\hbox{$#1$}%
               \setbox2=\hbox{$#2$}%
               \ifdim 1\wd1<1\wd2 % 
                    { \rlap{\mbox{\kern0.5\wd2\kern-0.5\wd1 \box1}} \box2 }
              \else {   \rlap{\mbox{$#1$}}
                        \rlap{\mbox{\kern0.5\wd1\kern-0.5\wd2 \box2}}
                        \phantom{#1} } 
              \fi }}

\def\rbox{\boxminus}
\def\rdia{\overstrike{\Diamond}{-}}
\def\cdia{\overstrike{\Diamond}{\rule{0.04em}{1.5ex}}}
\def\cbox{\inbox{\rule{0.04em}{1.6ex}}}
\def\cbox{\mathop{\setbox0=\hbox{$\square$}
                  \overstrike{\square}{\hbox{\rule{0.04em}{1\ht0}}}
             }}

\def\diaplus{\overstrike{\Diamond}{+}}




%% Make a modal box with a symbol centred inside
\def\inbox#1{\mathop{\setbox0=\hbox{$\square$}%
                      \setbox1=\hbox{#1}%
                      \rlap{\hbox{$\square$}}
                      \rlap{\mbox{\kern0.5\wd0\kern-0.5\wd1%
                                  \box1}}%
                      \phantom{\square}}}

\def\ibox{\mathop{\square\mkern-10mu
                   \mbox{\raise0.45ex\hbox{\tiny\em i}}
                  \mkern7mu
                 }
         }
\def\cdiamond{\mathop{\Diamond\mkern-12.7mu
                   \mbox{\raise0.4ex\hbox{\tiny\em c}}
                  \mkern7mu
                 }
         }


\def\s5box{\mathop{\square\mkern-10mu
                   \mbox{\raise0.45ex\hbox{\tiny 5}}
                  \mkern7mu
                 }
         }

\def\Fbox{\mathop{\square\mkern-11mu
                   \mbox{\raise0.45ex\hbox{\tiny \bf F}}
                  \mkern2mu
                 }
         }

\def\Box{\mathop\square}
\def\Diamond{\mathop\lozenge}

\def\bigI{\mathop{%
\hbox{%
%%
\def\thickness{1pt}%
\def\width{6pt}%
\def\height{0.6em}%
\def\gap{0.15em}%
%%
\dimen0=\thickness%
\divide\dimen0 by 2%
\dimen1=\width%
\divide\dimen1 by 2%
%%
\kern\gap%
\rlap{\vrule width\width height\thickness depth0pt}%
\rlap{\kern\dimen1 \kern-\dimen0%
\vrule width\thickness height\height depth0pt}%
\raise\height\hbox{\vrule width\width height\thickness depth0pt}%
\kern\gap}}}

\def\mconv{\mathop{% 
      \mbox{\raise0.35ex\hbox{$\scriptstyle\bigcirc$}}
                  }
          }
\def\mpconv{\circleddash}



% \def\putat#1#2#3{\setbox1=\hbox{#3}%
% \rule{0pt}{0pt}%
% \vspace*{#2}%
% \hspace*{#1}%
% \rule{0pt}{0pt}%
% \hbox to 0em{\rlap{\vtop to 0ex{\hbox to 0em{#3}}}}%
% \hspace*{-#1}\vspace*{-#2}}
% %\rule{0pt}{0pt}
% %\vspace*{-\ht1}\rule{0pt}{0pt}
% %\vspace*{-\ht1}

\def\putat#1#2#3{\rlap{\hbox{%
\kern#1% 
\vtop{\hbox{}%
\hbox{\vbox to #2{}} \hbox{#3}}}}}

% use putat within \hbox
% Objects will be located relative to baseline
% at start of the \hbox
% eg:
% \hbox{
% \putat{2in}{2in}{xxxxx}%
% \putat{2in}{2in}{00000}%
% \putat{5in}{3in}{Z}%
% \putat{2in}{2in}{0}%
% \putat{5in}{4in}{x}%
% }


%\long\def\putat#1#2#3{




\def\ri{\mathclose{\raise1ex\hbox{$\smile$}}}

\def\eqtag#1{\eqno ({\bf #1})}

\def\mc:#1{\hbox{${\mathcal{#1}}$}}
\def\mb#1{{\mathbf{#1}}}
 
\def\ifff{\qquad \hbox{if and only if} \qquad}

%%\def\d{\delta}

\def\QED{$\blacksquare$}

\def\epsrcgrant{GR/K65041}
\def\VUGgrant{GR/M56807}

\def\Boxx{\boxtimes}

%% Force a math symbol to go into scriptsize
\def\mscriptsize#1{\hbox{\scriptsize $#1$}}
\def\mlarge#1{\hbox{\large $#1$}}

\def\boxtheorem#1#2{~\\
\centerline{\fbox{
\parbox{5in}{
\centerline{\bf #1}
#2
}}}
\vspace{1ex}}

\def\proof#1#2{
\begin{quotation}
\noindent {\bf Proof of #1:}
#2
\QED
\end{quotation}
}

\def\proofpar{\\ \hspace*{1.5em}}


%%%% ENVIRONMENTS

%% specify my list parameters
%% These can be overridden by declarations within
%% a document.

\def\mytopsep{1ex}
\def\mypartopsep{0ex}
\def\myitemsep{0.3ex}
\def\myparsep{0ex}
\def\mylistparindent{2em}
\def\myleftmargin{4.5em}
\def\mylabelwidth{4.5em}
\def\mylabelsep{1em}
\def\myitemindent{0em}

\def\setmylistparams{%
             \setlength{\topsep}{\mytopsep}
             \setlength{\partopsep}{\mypartopsep}
             \setlength{\itemsep}{\myitemsep}
             \setlength{\parsep}{\myparsep}
             \setlength{\listparindent}{\mylistparindent}
             \setlength{\leftmargin}{\myleftmargin}
             \setlength{\labelwidth}{\mylabelwidth}
             \setlength{\labelsep}{\mylabelsep}
             \setlength{\itemindent}{\myitemindent}
        }

\def\clistbullet{$\bullet$}

\newcounter{tempvalue}
% A compact list making environment
\newenvironment{clist}
    { %\vspace{1ex}
      \begin{list}
            {\labelitemi}
            \setmylistparams
    }
    { \end{list}\addvspace{1ex} 
    }
% A compact list making environment
% with -- as the default item tag
\newenvironment{cdashlist}
    { %\vspace{1ex}
      \begin{list}
            {--}
            {\setlength{\topsep}{0.2ex}
             \setlength{\partopsep}{0ex}
             \setlength{\itemsep}{0.3ex}
             \setlength{\parsep}{0ex}
             \setlength{\listparindent}{2em}
            }
    }
    { \end{list}\addvspace{1ex} 
    }


% compact list with wide labels for item tags
\newenvironment{taglist}[1]
    { %\vspace{1ex}
      \begin{list}
            {$\bullet$}
            {\setlength{\topsep}{0.2ex}
             \setlength{\partopsep}{0ex}
             \setlength{\itemsep}{0.3ex}
             \setlength{\parsep}{0ex}
             \setlength{\listparindent}{2em}
             \setlength{\leftmargin}{#1}
             \setlength{\labelwidth}{#1}
             %\addtolength{\labelwidth}{-1em}
             \setlength{\labelsep}{0em}
            }
    }
    { \end{list}\addvspace{1ex} 
    }

\newcounter{cenumcount}
\newenvironment{cenum}
    { %\vspace{1ex}
      \begin{list}
            {\arabic{cenumcount}.}
            {\usecounter{cenumcount}
               \setmylistparams
            }
    }
    { \end{list}\addvspace{1ex} 
    }

\newcounter{romcount}
\newenvironment{romanenum}
          { 
            \begin{list}{\roman{romcount})~~}
                        {\usecounter{romcount}
                         \setmylistparams
                        }
          }
          { \end{list} }

\newcounter{alphcount}
\newenvironment{alphenum}
          { \begin{list}{\alph{alphcount})~~}
                        {\usecounter{alphcount}}}
          { \end{list} }


%% Labelled list environment
\newenvironment{llist}[1]{%
\newcounter{#1}
\begin{list}{({\bf #1\arabic{#1}})\hspace{1em}}{\usecounter{#1}}
}
{\end{list}}

%% Labelled list continued
%% (assumes counter already exists)
\newenvironment{llistcont}[1]{%
\setcounter{tempvalue}{\value{#1}}
\begin{list}{({\bf #1\arabic{#1}})\hspace{1em}}{%
\usecounter{#1}
\setcounter{#1}{\value{tempvalue}}
}
}
{\end{list}}

%%% compact versions of llist and llistcont
\newenvironment{cllist}[1]{%
\newcounter{#1}
\begin{list}{({\bf #1\arabic{#1}})}
             { \usecounter{#1}
               \setmylistparams
             }
}
{\end{list}}

%% Labelled list continued
%% (assumes counter already exists)
\newenvironment{cllistcont}[1]{%
\setcounter{tempvalue}{\value{#1}}
\begin{list}{({\bf #1\arabic{#1}})}{%
                 \usecounter{#1}
                 \setcounter{#1}{\value{tempvalue}}
                \setmylistparams
          }
}
{\end{list}}


\def\longdef{\long\def}

%% Define a labelled list type
%% #1 name of control sequence for the list type
%% #2 prefix for numbering
\def\llisttype#1#2{%
\newcounter{#2}
\expandafter\longdef\csname #1list\endcsname##1{%
\begin{cllistcont}{#2}
##1
\end{cllistcont}}
\expandafter\def\csname #1\endcsname##1{%
\begin{cllistcont}{#2}
\item
##1
\end{cllistcont}}
}

\def\litem#1#2{%
\begin{clist}
\item[({\bf #1})] 
#2
\end{clist}}

\newenvironment{chapabst}%
{\begin{quotation}\small\noindent }{\end{quotation}}

% Split line displayed formulae
% (not actually an environment)
\newcommand{\splitdismath}[4]{%
{\setlength{\arraycolsep}{0em}
\begin{eqnarray}
& #1 & #2 \nonumber \\
 &    & #3 
\label{#4}
\end{eqnarray}
}
}

% Displayed equation with tabbing

\newcommand{\tabmathn}[2]
{ \begin{equation}
\vbox{
\vspace{-2ex}
\begin{tabbing}
#1
\end{tabbing}
\vspace{-3.5ex}
}
\label{#2}
\end{equation}
}

\newcommand{\displaytab}[1]
{ \begin{displaymath}
\vbox{
\vspace{-2ex}
\begin{tabbing}
#1
\end{tabbing}
\vspace{-3.5ex}
}
\end{displaymath}
}

\def\myeq#1{$$\eqalign{#1}$$}


%%% Reinstate \eqalign macros removed by Lamport
\catcode`\@=11
\def\eqalign#1{\null \,\vcenter {\openup \jot \m@th 
\ialign {\strut \hfil $\displaystyle{##}$&$\displaystyle {{}##}$\hfil
 \crcr #1\crcr }}\,}

\def\eqalignno#1{\displ@y \tabskip \centering 
\halign to\displaywidth {\hfil $\@lign \displaystyle{##}$\tabskip \z@skip
 &$\@lign \displaystyle {{}##}$\hfil \tabskip \centering 
 &\llap {$\@lign ##$}\tabskip \z@skip \crcr 
 #1\crcr }}

\def\leqalignno#1{\displ@y \tabskip \centering
 \halign to\displaywidth {\hfil $\@lign \displaystyle {##}$\tabskip \z@skip
 &$\@lign \displaystyle {{}##}$\hfil \tabskip \centering
 &\kern -\displaywidth \rlap {$\@lign ##$}\tabskip \displaywidth \crcr 
 #1\crcr}}

%%% Define something like \bibliography but only records the
%%% bibfiles (does not input the bbl file)
\def\bibliographyfiles#1{%
  \if@filesw
    \immediate\write\@auxout{\string\bibdata{#1}}%
  \fi}

%%% Define a separate command to input the bbl file.
\def\bibliographytext{%
  \@input@{\jobname.bbl}}


%% Suppress printing of bibitems in thebibliography
%% This is for use with the `inlinebib' package,
%% which adds fullcitations where you give the \cite command.
%% Doesn't print note
%% Could add note by changing ##2 to ##2##3 in the \bibcite argument
%% But output not quite right.
\def\suppressendbib{%
\long\def\@bibitem##1 ##2 \note ##3 \short ##4 \end{\par\if@filesw
        {\def\protect####1{\string ####1\space}%
         \let\newblock\@empty
         \immediate\write\@auxout{\string
                \bibcite{##1}{\string\@bibcall{##1}{##2}{##4}}}}\fi
        }
\def\thebibliography##1{}
}


%% Change indent space allocated by \thebibliography
%% use before \bibliography or \bibliographytext 
\def\setbibindent#1{%
\let\oldthebibliography\thebibliography
\def\thebibliography##1{\oldthebibliography{#1}}
}



\catcode`\@=12

\long\def\skipover#1{}

\long\def\correction#1{\footnote{TO CORRECT: #1}}
\def\nocorrections{\long\def\correction##1{}}
\def\question#1{{\bf Question:} {\em #1} }
\long\def\todo#1{{\bf To Do:} {\em #1} }


\def\twiddle{$\sim$}

%% Macro for URLs
%% puts them in tt format and  breaks line after dot if needed.

\skipover{
\def\urlspace{-0.5em}
\def\url#1{{\tt \urlsub#1\end}}
%\def\urldot{. \hspace{-1em}\urlsub}
\def\urldot{. \linebreak[1]\hspace\urlspace\urlsub}
\def\urlcolon{: \linebreak[1]\hspace\urlspace\hspace\urlspace\urlsub}
\def\urlslash{/ \linebreak[1]\hspace\urlspace\urlsub}
\def\urltilde{\twiddle\urlsub}
\def\urlsub#1{\ifx#1\end \let\next=\relax
      \else \ifx#1.\let\next=\urldot
      \else \ifx#1:\let\next=\urlcolon
      \else \ifx#1/\let\next=\urlslash
      \else \ifx#1~\let\next=\urltilde
      \else#1\let\next=\urlsub\fi\fi\fi\fi\fi\next}
}

% Input a file in verbatim mode (eg a program file)
\def\verbinput#1{\expandafter\begin{verbatim}\input#1}
%NB: usage: \verbinput{filename}\end{verbatim}


\def\mb#1{{\mathbf{#1}}}
\def\mbf#1{{\mathbf{#1}}}


%%% MACROS for SLIDES


\def\heading#1{%
\centerline{\shadowbox{
\large \begin{tabular}{c} #1 \end{tabular}
           }  }
\vspace{1ex minus 1ex}
}

\def\bheading#1{%
\centerline{\shadowbox{
\large\bf \begin{tabular}{c} #1 \end{tabular}
           }  }
\vspace{1ex minus 1ex}
}

\def\printlandscape{\special{landscape}}    % Works with dvips.
\def\printportrait{\special{portrait}}

\def\leedslogo#1{\centerline{
\psfig{file=uol.eps,width=#1} %% May look wrong in xdvi
}}

\def\titleslide#1#2{
\begin{slide*}
~

\vfill
\heading{#1}

\vfill
\begin{center}
        {\large\bf Brandon Bennett} 

\vspace{3ex}
        Division of AI \\
        School of Computing \\
        University of Leeds\\ 
        Leeds LS2 9JT, England \\
        {\tt brandon@comp.leeds.ac.uk} \\
\end{center}

\vfill
\centerline{
\psfig{file=uol.eps,width=0.7in} %% May look wrong in xdvi
}
\vfill
\centerline{#2}
\vfill
\end{slide*}
}


\def\nlf{\\ \mbox{} \hfill}

%%\def\fixhyphens{\usepackage[english]{babel}}

%% FIX HYPHENATION
\lefthyphenmin=2
\righthyphenmin=3

%% special commands for spell checker
\def\sic#1{#1}
\def\sico!#1!{#1}
\def\sicname#1{#1}
\def\accept#1{}
\def\acceptname#1{}

%\newenvironment{nospell}{}{}
\def\spelloff{}
\def\spellon{}


%% Check whether a macro is defined
\def\ifundefined#1{\expandafter\ifx\csname#1\endcsname\relax}

%% Define \ifpdf
%% Used in graphic stuff to check if pdf is running
%% This is taken from ifpdf.sty
\newif\ifpdf
\ifx\pdfoutput\undefined
\else
  \ifx\pdfoutput\relax
  \else
    \ifcase\pdfoutput
    \else
      \pdftrue
    \fi
  \fi
\fi

\ifpdf
  \typeout{Running in PDF mode}
\else
  \typeout{Running in DVI mode}
\fi

%%%% CLEVER GRAPHIC INCLUSION

\def\pdfgraphictype{jpg} %% or perhaps pdf or png
\ifpdf   
  \def\graftype{\pdfgraphictype}  
\else
  \def\graftype{eps}
\fi


\typeout{* BB mymacros: Default graphic type set to: \graftype}

\ifpdf
   \def\optpdftex{pdftex}
   \def\optgraphic[#1]#2#3{\includegraphics[#1]{#3}} 
\else
   \def\pdfinfo#1{}
   \def\optpdftex{}
   \def\optgraphic[#1]#2#3{\includegraphics[#1]{#2}} 
\fi

\def\altgraphic[#1]#2#3#4{\optgraphic[#1]{#2.#3}{#2.#4}}

\def\figurepath{}
\def\graphswap[#1]#2{\optgraphic[#1]{\figurepath#2.eps}{\figurepath#2.\pdfgraphictype}}

%% \graphicifpdf includes graphic in pdf mode
%% otherwise just draws a box with the file name
%% I haven't really used this much.
\ifpdf
     \def\graphicifpdf[#1]#2{\includegraphics[#1]{#2}}
   \else
     \def\graphicifpdf[#1]#2{\fbox{\parbox{1in}{#1\\ #2}}}
\fi 

%\def\setaltgraphic#1#2{%
%   \def\altgraphic[##1]##2{\optgraphic[##1]{##2.#1}{##2.#1}}}

\def\graphic[#1]#2{\includegraphics[#1]{#2.\graftype}} 

\def\psfiggraphic{\renewcommand{\graphic}[2][]{\psfig{##1,file=##2.eps}}}

%%% Date functions

\def\dayth{\number\day\ifcase\day \or
st\or nd\or rd\or th\or th\or th\or th\or th\or th\or th\or 
th\or th\or th\or th\or th\or th\or th\or th\or th\or th\or
st\or nd\or rd\or th\or th\or th\or th\or th\or th\or th\or
st\fi}

\def\monthname{\ifcase\month \or
January\or February\or March\or April\or May\or June\or
July\or August\or September\or October\or November\or December\fi}


%%%% How to set up new font sizes
%% define a font which is Helvetica at 4.5 pt
%\newfont{\boxfont}{helvetica at 4.5 pt}

% How to rename or alter the References section
% \renewcommand\refname{References\footnote{The   reference  list  given
%     here is somewhat incomplete due to limited preparation time. It is
%     intended  to  add several  more  recent  references  to the  final
%     version of this paper, relating the discussion to current works on
%     ontology      development.}}      


\typeout{** finished loading mymacros.tex **}

%%%%%%%%%%%%%%%%%%%%%%%%%%%%%%%%%%%%%%%%%%%%%%%%%%%%%%%%%%%%%%%
%%  END END END         (of mymacros.tex)      END END END   %%
%%%%%%%%%%%%%%%%%%%%%%%%%%%%%%%%%%%%%%%%%%%%%%%%%%%%%%%%%%%%%%%

\usepackage{url}


\ifx\pdfoutput\undefined
% we are running LaTeX, not pdflatex
\usepackage{graphicx}
\else
% we are running pdflatex, so convert .eps files to .pdf
%\usepackage[pdftex]{graphicx}
%\usepackage{epstopdf}
\fi 

\copyrightyear{}
\pubyear{}

\def\partOf{\pr{part\_of}}
\def\hasPart{\pr{has\_part}}
\def\isA{\pr{is\_a}}
\def\instanceOf{\pr{instance\_of}}
\def\derivesFrom{\pr{derives\_from}}
\def\adjacentTo{\pr{adjacent\_to}}

\def\partOfAtSomeTimes{\pr{part-of-at-some-times}}
\def\partOfAtAllTimes{\pr{part-of-at-all-times}}
\def\hasPartAtSomeTimes{\pr{has-part-at-some-times}}
\def\hasPartAtAllTimes{\pr{has-part-at-all-times}}
\def\hasPartAtAllTimesForWhichPartExists{\pr{has-part-} \pr{at-all-times-}\ \pr{for-which-part-exists}}
\def\partOfAtAllTimesForWhichWholeExists{\pr{part-of-} \pr{at-all-times-}\ \pr{for-which-whole-exists}}

\def\atAllTimes{\pr{at-all-times}}
\def\atSomeTimes{\pr{at-some-times}}
\def\atAllTimesForWhichSubjectExists{\pr{at-all-times-for-which-subject-exists}}

\def\CellNucleus{\pr{cell nucleus}}
\def\Cell{\pr{cell}}

\def\OBOREL{\textbf{OBO-REL}}

\newcommand{\tbleqn}[1]{
\begin{math}
\begin{aligned}[1]
#1
\end{aligned}
\end{math}
}

\begin{document}
\firstpage{1}

\title{A critique of rigid temporalized relations}

\author{Christopher J. Mungall\,$^{1}$\footnote{to whom correspondence should be addressed}}
\address{$^{1}$Genomics Division, Lawrence Berkeley National Laboratory, MS84R017, 1 Cyclotron Road, Berkeley, CA 94720 USA}

\history{}

\editor{}

\maketitle

\begin{abstract}

  In this report I evaluate the proposed new temporalized relations
  strategy in which many existing relations would be replaced by two
  ore more relations, an \emph{at-all-times}\ and an \emph{at-some-times}\
  form.

  My findings are that the \emph{at-all-times}\ relations have an
  underlying logical problem that renders them formally incorrect for
  use in many ontologies that represent structures that change over
  time. The \emph{at-some-times} relations are safer, but would lose
  crucial transitive inferences. These logical problems are compounded
  by the fact that the relations are difficult for users to
  understand, and will most likely lead to confusion and errors,
  especially in the absence of detailed documentation.

  I conclude that these relations should not be adopted by ontology
  editors. Migrating to these relations would be an expensive,
  error-prone process that would alienate the users of ontologies, and
  the end result would be ontologies that are either formally
  incorrect or too weak to perform required inferences.

\end{abstract}

\section{Introduction}

The OBO relations ontology (\OBOREL) defined a set of core relations
for use in biological ontologies, including \isA, \partOf\ and
\derivesFrom\cite{Smith2005}. The original \OBOREL\ paper has been
cited 709 times\footnote{Google scholar}, and has been a crucial
reference in the correct usage of relations in biological ontologies.

One notable aspect of \OBOREL\ was the precise specifications of how
relationships change or remain the same through the passage of
time. For example, a user of \OBOREL\ could say that \emph{every}
\CellNucleus\ is part of \emph{some} \Cell\ at \emph{any} given moment
of time (for which that cell nucleus exists).

$
\A x \A t : \instanceOf(x, \CellNucleus, t) \imp \\
 \E y \instanceOf(y, \Cell, t), \partOf(x,y,t)
$

This relationship is known as \emph{permanent generic parthood}.

The other notable aspect of \OBOREL\ was the distinction between type
level and instance level relations. Each type-level relation connects
a pair of classes and typically is defined according to an
\pr{ALL-SOME-ALLTIMES}\ pattern - for example, every cell nucleus is
part of some cell at all times. The use of type level relations causes
some confusion when using OWL, which does not support the same
mechanism. Instead the relationship between a nucleus and a cell in
OWL is explicitly quantified, but without a time argument, as all
relations in OWL are binary: For example, \emph{every cell nucleus is
  part of some cell}. As of 2010, the official semantics of OBO format
have been as a subset of OWL, so what applies to OWL necessarily
applies to OBO.

This impedance mismatch between the OWL interpretation and the
\OBOREL\ account has been a problem is providing a consistent formal
account of relations, although it is not clear that this has caused a
problem for many ontology developers or users. The standard approach
has been to use a set of binary relations as specified in Table
\ref{tab:characteristics-atemporal}.

As part of the release process for the OWL translation of version 2 of
the Basic Formal Ontology (BFO)\cite{Grenon2004}, a number of people
explored different strategies for unifying OWL binary properties with
the ternary relations in the BFO2 reference
specification\cite{Graz}. One such strategy is the \emph{temporalized
  relations} strategy, in which each reference relation relating
continuants has two or more OWL cognates, \pr{rel-at-some-times} and
\pr{rel-at-all-times}.

In this review I do not attempt to compare or even describe the
different modeling possibilities, instead I focus purely on the
``temporalized relations'' strategy. I first provide an outline of
temporalized relations, drawing on the existing release notes,
attempting to fill some gaps. I then present the major problems posed
by these relations: (1) the relations fail to capture the biological
reality, forcing ontology editors to make a tradeoff between two
unsatisfactory choices (2) the relations are confusing for users and
even experienced ontology editors. These problems are related, in that
ontology editors may not understand the tradeoff they are being asked
to make.

Finally I end with some specific recommendations regarding the
temporalized relation strategy in general and BFO2 specifically.


\section{Temporalized Relations}

Here I distinguish between reference relations (RRs) and their
manifestation in OWL as binary temporalized relations (TRs), using the
temporalized relation strategy (TRS). All relations are instance
level. As a typographic convention I use dashes to separated the words
in a temporalized relation, and underscores in a reference relation.

The BFO2 Graz version release notes\cite{Graz} specify a general
template for relating RRs to TRs:

\begin{verbatim}
x rel-at-some-time y ->
 exists(t) exists_at(x,t) -> 
    exists_at(y,t) and rel(x,y,t)
x rel-at-all-times y ->
  forall(t)  exists_at(x,t) -> 
      exists_at(y,t) and rel(x,y,t)
\end{verbatim}

Here we focus on \partOf\ as an exemplar relation, whilst recognizing
that similar patterns may apply to other, but not all relations.

For \partOf\ connecting two continuants there are three TRs:

\begin{enumerate}

\item \partOfAtSomeTimes

\item \partOfAtAllTimes

\item \partOfAtAllTimesForWhichWholeExists

\end{enumerate}

Note that in BFO2 the actual label is ``part of continuant at some/all
times'', we shorten this for brevity.

On the surface, the \atAllTimes\ form appears to be the same as the
\OBOREL\ interpretation. For example, the OWL axiom:

$$
\CellNucleus\ \pr{SubClassOf}\ \partOfAtAllTimes\ \pr{some}\ \Cell
$$

may seem to be the same as the statement ``every cell nucleus is part
of some cell at all times''. We might even be able to automatically
translate an ontology written using the \OBOREL\ interpretation into
TRs. \emph{However, these are NOT the same, and an understanding of
  why is crucial to correct usage of these relations and an
  understanding of the consequences of using them}.

SHOW FOL HERE

The fact that these are different is of utmost importance to how
ontologies are created, and affects the \emph{characteristics} of
these relations in some ways that might seem surprising.

\subsection{Object property characteristics}

In OWL, relations (object properties) can have certain characteristics
such as \emph{transitive}, \emph{symmetrical}, and they may have
additional logical axioms such as \emph{inverse properties}. These are
extremely useful for many purposes - transitivity has been at the core
of bioinformatics applications of ontologies from the initial version
of the Gene Ontology\cite{Ashburner2000}, if not before. Inverse
properties are useful for instance level reasoning, and for finding
errors in complex ontologies.

When translating a RR to a TR, it may not be immediately clear what
properties of the RR should be carried over to the TR. Ideally we
would be able to prove that the RR and TR are consistent, although the
proof may be obvious to an expert logician. Here I use \partOf\ as an
illustrative example, although each relation may require individual
examination for its properties.

Table \ref{tab:characteristics-temporalized} shows some of the
property characteristics of the various forms of the continuant
parthood relations.

\subsubsection{Transitivity}

In the case of the RR \partOf\ (which is transitive), the stronger
\atAllTimes\ TR retains the transitivity characteristic, whereas the
weaker \atSomeTimes\ TR does not have this. This means the weaker
version is often safer to use in an ontology, but will lead to fewer
inferences.

\subsubsection{Symmetricality}

For other relations and other characteristics, the translation may not
be obvious. For example, the adjacency relation is commonly assumed to
be symmetric\footnote{we are only considering instance level
  relations}. Should this symmetricality characteristic be carried
over to the temporalized form?

Currently BFO2 does not have an adjacency relation, or any other
reference relation that is symmetric, so the following is based on my
own understanding. I would assume that the symmetricality should be
declared for the \atSomeTimes\ form and not the \atAllTimes\ form. In
contrast to the partOf and transitivity case, here it is the weaker
form of relation that inherits the characteristic.

See table \ref{tab:characteristics-adj} for a summary.

\subsubsection{Subproperties}

We assume the following hierarchy:


$$ \atSomeTimes < \atAllTimesForWhichSubjectExists < \atAllTimes $$

Note how this interacts with other properties. If $x \adjacentTo\
\atAllTimes y$, then we can infer that $y \adjacentTo \atSomeTimes x$.

\subsubsection{Inverse Properties}

Ontologies frequently declare inverse relations. For example, the
RR \partOf\ is the inverse of \hasPart. This is fairly standard
practice, and the inverse relations are extremely useful for
reasoning. In OWL, it's not strictly necesary to declare an inverse,
as it is possible to use an InversePropertyExpression. Here we take
the view that inverses that have typically been declared in previous
ontologies are useful, and should also be declared as RRs and have
corresponding TRs.

The inverse of the TRs of \partOf\ may not be completely
intuitive. Naively we might guess that \partOfAtAllTimes\ would be the
inverse of \hasPartAtAllTimes, but this not the case.

In fact, the declaring the inverse of \partOfAtAllTimes\ requires
declaring a third TR form, \hasPartAtAllTimesForWhichPartExists. The
\atAllTimesForWhichSubjectExists\ form can be generated for some but
not all RRs. Whilst it complicates the ontology of TRs to introduce
this extra form, the alternative of not having inverses (or of being
forced to write complicated inverse expressions) may be too
prohibitive for many users.




\subsubsection{Other characteristics}

We do not consider other characteristics such as antisymmetry and
domain/range restrictions here.


\subsubsection{Generating a TR from an RR}

Currently there is no ``recipe'' for generating a set of TRs from an
RR. Different patterns may apply to different RRs.

For example, when creating the TRs for inheresIn and its inverse
bearerOf, it is correct to declare bearerOfAtAllTimes InverseOf
inheresInAtAllTimes.


\begin{table}
\begin{tabular}{ | p{3cm} | p{1cm} | p{1cm} | p{3cm} | }
\hline
\textbf{Relation} & \textbf{Trans} & \textbf{Symm} & \textbf{Inverse Of}  \\
\hline
\partOf\  & Yes & No & \hasPart  \\
\hline
\hasPart\ & Yes & No & \partOf \\
\hline
\adjacentTo\ & No & Yes &  \\
\hline
\end{tabular}
\caption{Relation characteristics, atemporal. These are the characteristics of the instance level relations in the current RO}
\label{tab:characteristics-atemporal}
\end{table}


\begin{table}
\begin{tabular}{ | p{3cm} | p{1cm} | p{4cm} | }
\hline
\textbf{Relation} & \textbf{Trans} & \textbf{Inverse Of}  \\
\hline
\partOf\ \atSomeTimes & No & \hasPart\ \atSomeTimes \\
\hline
\partOf\ \atAllTimes & Yes & \hasPartAtAllTimesForWhichPartExists \\
\hline
\partOfAtAllTimesForWhichWholeExists & Yes & \hasPart\ \atAllTimes \\
\hline
\hasPart\ \atSomeTimes & No & \partOf\ \atSomeTimes \\
\hline
\hasPart\ \atAllTimes & Yes & \partOfAtAllTimesForWhichWholeExists \\
\hline
\hasPartAtAllTimesForWhichPartExists & Yes & \partOf\ \atAllTimes \\
\hline
\end{tabular}
\caption{Relation characteristics for core continuant parthood relations. These characteristics are declared in the current BFO2 OWL Graz version}
\label{tab:characteristics-temporalized}
\end{table}


\begin{table}
\begin{tabular}{ | p{3cm} | p{1cm} | p{4cm} | }
\hline
\textbf{Relation} & \textbf{Symm} & \textbf{Inverse Of}  \\
\hline
\adjacentTo\ \atSomeTimes & Yes & \adjacentTo\ \atSomeTimes \\
\hline
\adjacentTo\ \atAllTimes & No & \\
\hline
\end{tabular}
\caption{Relation characteristics for a typical symmetric relation. This relation is not part of the BFO2 OWL Graz version}
\label{tab:characteristics-adj}
\end{table}


\begin{table}
\begin{tabular}{ | p{1.8cm} | p{6.2cm} | }
\hline
\textbf{Relation} & \textbf{Axiom}  \\
\hline
\partOf &
        \tbleqn{
 & \A x \A t : \instanceOf(x, \CellNucleus, t) \imp \\
 & \E y : \instanceOf(y, \Cell, t), \partOf(x,y,t)
} \\
\hline
\pr{part-of-} \atAllTimes &
        \tbleqn{
 & \A x : \instanceOf(x, \CellNucleus) \imp \\
 & \E y \instanceOf(y, \Cell), \partOf(x,y,t)
} \\
\hline
\pr{part-of-} \atSomeTimes &
        \tbleqn{
 & \A x \E t : \instanceOf(x, \CellNucleus, t) \imp \\
 & \E y \instanceOf(y, \Cell, t), \partOf(x,y,t)
} \\
\hline
\end{tabular}
\caption{Summary of relations. }
\label{tab:relation-table}
\end{table}

\section{Evaluation}

\subsection{Temporalized Relations do not reflect the intentions of
  ontology editors}

TRs present many challenging problems. For example, when converting an
anatomy ontology that has been modeled traditionally using the
relations in table \ref{tab:characteristics-atemporal}, the ontology
editor must make a choice on a per-relationship basis as to which of
the relations in \ref{tab:characteristics-temporalized} should be
used.

This is an onerous task, but this could be justified if the results
were better ontologies. However, in many cases \emph{none of the
  choices are appropriate}.

This is because for many ontologies, the most appropriate choice of
parthood relationship is the \emph{permanent-generic} form, as
specified in the original OBO relations paper. The standard example
here is the relationship between a cell nucleus and a cell. At any
moment in time, a given cell nucleus is by definition part of some
cell\footnote{we would consider extruded nuclei to be transformations
  of cell nuclei, but instantiating a different class}. However,
\emph{this need not be the same call throughout the lifetime of the
  nucleus}.

In contrast, if the \partOfAtAllTimes\ TR is used then the
interpretation is that the cell nucleus is always part of the same
cell. This interpretation can be proved to be formally wrong in cells
that undergo cell division\ref{CellDiv}. This is because the cell
nucleus is always part of a cell, \emph{but not the same cell}.

The ontology editor can not then choose to use \partOfAtAllTimes\
without making a false statement. They may then decide to
use \partOfAtSomeTimes. Such a usage would be formally valid, but
incomplete from the point of view of useful reasoning. This is because
the weaker \atSomeTimes\ form lacks the transitivity characteristic.

A third possibility is to use
the \partOfAtAllTimesForWhichWholeExists\ form, but this would also be
false (note the above proof needs to be extended to cover this case).

The ontology editor requires the permanent generic form in order to be
both accurate and to get the required inferences. However, this form
is specifically excluded in the TR strategy.

The cell nucleus example is the standard one, because it is central to
all of eukaryotic biology. It is by no means the only such
example. The problem arises whenever we have material passed around
from one carrier to another.  Enumerating a list of examples is
difficult because the instance level identity conditions may not be
clear. 

the problem is not limited to parthood relations. Use of TRs requires
that all continuant relations are temporalized. This includes
relations used to classify structures by phenotype.

\subsection{Rigidity requirement is too onerous}

The BFO2 Graz release notes state:

\begin{quotation}

  Thus we only instantiate ``rigid'' classes, as the interpretation we
  take is a rdf:type C => forall(t) a exists at t -> a instance of C
  at t. Temporally restricted instantiation is not supported in this
  version of BFO in OWL. We are working on it for the future.

\end{quotation}

To many users this may seem like an obscure point, but it is actually
quite a sever restriction. A class is rigid if it is instantiated
``for life''. If an individual transforms from being an instance of
one class of thing to another, then those classes are not rigid.

An example of a rigid class may be ``Homo sapiens''. If an individual
instantiates this class at some time t, then they instantiate it all
times for which they exist (barring some unusual inter-species
transformation).

This clause means material entity classes such as the following may
not be supported in this version of BFO:

\begin{enumerate}

  \item 'human with Parkinson's disease'
  \item 'female organism'
  \item 'infected lung'
  \item 'professor' (but \emph{professor role} is allowed)
  \item 'human patient' (but \emph{patient role} is allowed)
  \item 'oocyte'
  \item 'fractured bone'
  \item 'happy human'
  \item 'fetal heart'
  \item 'neural crest cell'
  \item 'open heart valve'
  \item 'gravid uterus'
  \item 'phosphorylated EGFR protein'
  \item 'cytoplasmic NFkB'
  \item any leaf node from PATO

\end{enumerate}

In some cases the constraint may not be so onerous. It can be argued
that a well structured ontology would never include a class
``professor'', and that this should always be modeled using a rigid
class (human) plus a role (professor role). However, these decisions
should be made on a case by case basis by each ontology that than
imposed from the upper ontology.

In other cases the distinction between rigid and non-rigid may not be
clear. One can argue that when an EGFR protein changes state from
being unphosphorylated to a phosphorylated state it is no longer the
same instance - the protein literally ceases to exist and is replaced
by a distinct individual an instant later, sharing all the same
properties except that it is phosphorylated. In fact one could take
this position for all of the above cases, in which case the TR
strategy becomes similar to the temporally qualified continuant (TQC)
strategy. I do not explore this further, as I assume this is contrary
to the expectations of the TR proponents.

When considering BFO2 in particular, there is an interesting
disjunction between the reference document, which explicitly states
that \emph{determinates} (for example, qualities such as ``square'',
``charged'', ``cylindrical'') are non-rigid. The Graz release states
that instantiation of these classes is not supported. These two
seemingly contradictory statements are not explicitly linked
anywhere. The modeling implications of this disconnect are not clear,
and require further documentation. It cannot be ruled out that this
restriction will involve further complexity.

It may be the case that future versions of the TR strategy will allow
for non-rigid classes. It is not clear how this will be achieved
without additional complexity.  The TR strategy must be evaluated on
what exists presently, and at this time the strategy comes with
constraints that ontology developers should be fully aware of.

\subsection{Temporalized relations add complexity}

The most striking feature of an ontology that uses the TR strategy is
the complexity. Whereas using traditional modeling, we may have has a
single parthood relation, we now have three.

In theory some of this complexity could be tamed by additional tooling
(although it is not clear who has the resources to implement
this). However, even if this can be hidden from the user, the ontology
developer is forced to wrestle with the complexity.

This complexity first manifests when an ontology developer chooses to
migrate from a traditionally modeled ontology using relations from
table \ref{tab:characteristics-atemporal}, assuming an \OBOREL\
interpretation.

For each axiom that uses a relation that has multiple variants in TR
form, the developer must make a choice of which one to use....

BAD SMELL

\section{Discussion}

\subsection{Recommendations}

\subsubsection{Do not use} My primary recommendation is that
Temporalized Relations should not be used in their current
form. Ontologies should not migrate to them.

\subsubsection{Documentation} The TR strategy needs much more
documentation if ontology developers are to use TRs. Even if TRs are
abandoned in their current form (as I recommend), documentation would
be useful to be able to help achieve consensus on this matter.

\subsubsection{Alternate strategies} Given the inherent limitations
and complexity of TRs, adequate consideration should be given to
alternate strategies such as Temporally Qualified Continuants
(TQCs). The ``null'' strategy of continuing to use simple OWL object
proeprties as if they has a \OBOREL\ interpretation should be the
default strategy until an adequate replacement is found.

\subsubsection{Use cases} If adopted, TRs will require tremendous
effort in omntology migration, documentation, tooling. There is little
to motivate ontology developers to do this as the existing strategy
works for them. The main motivating factor seems to be a desire for
formal correctness, at the expense of usability and biological
correctness.

\section{Conclusions}

Temporalized Relations would be a fundemental change to the way
relationships are modeled in ontologies. They would introduce
additional complexity.

Some of these costs could be justified if Temporalized Relations were
on a path towards making ontologies more biologically
accurate. However, due to the built in lack of support for permanent
generic parthood and non-rigid classes, migrating to Temporalized
Relations would lead to ontologies becoming \emph{less} accurate. My
recommendation is unambiguous in its rejection of the use of
Temporalized Relations in biological ontologies.

\section*{Acknowledgments}



%\section*{References}

% ========================================
\bibliography{trc}
\bibliographystyle{plain}
% ========================================

\section*{Appendix}

\begin{table}
\begin{tabular}{ | p{1.8cm} | p{6.2cm} | }
\hline
\textbf{Axiom}  \\
\hline
\partOf &
        \tbleqn{
 & \A x \A t : \instanceOf(x, \CellNucleus, t) \imp \\
 & \E y : \instanceOf(y, \Cell, t), \partOf(x,y,t)
} \\
\hline
\pr{part-of-} \atSomeTimes &
        \tbleqn{
 & \A x \E t : \instanceOf(x, \CellNucleus, t) \imp \\
 & \E y \instanceOf(y, \Cell, t), \partOf(x,y,t)
} \\
\hline
\end{tabular}
\caption{Summary of relations. }
\label{tab:relation-table}
\end{table}


\end{document}
