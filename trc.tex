\documentclass{bioinfo}
%\documentclass{article}
\input{logicmacros}
\usepackage{url}


\ifx\pdfoutput\undefined
% we are running LaTeX, not pdflatex
\usepackage{graphicx}
\else
% we are running pdflatex, so convert .eps files to .pdf
%\usepackage[pdftex]{graphicx}
%\usepackage{epstopdf}
\fi 

\copyrightyear{}
\pubyear{}

\def\partOf{\pr{part\_of}}
\def\hasPart{\pr{has\_part}}
\def\isA{\pr{is\_a}}
\def\instanceOf{\pr{instance\_of}}
\def\derivesFrom{\pr{derives\_from}}
\def\adjacentTo{\pr{adjacent\_to}}

\def\existsAt{\pr{exists\_at}}

\def\partOfAtSomeTimes{\pr{part-of-at-some-times}}
\def\partOfAtAllTimes{\pr{part-of-at-all-times}}
\def\hasPartAtSomeTimes{\pr{has-part-at-some-times}}
\def\hasPartAtAllTimes{\pr{has-part-at-all-times}}
\def\hasPartAtAllTimesForWhichPartExists{\pr{has-part-} \pr{at-all-times-}\ \pr{for-which-part-exists}}
\def\partOfAtAllTimesForWhichWholeExists{\pr{part-of-} \pr{at-all-times-}\ \pr{for-which-whole-exists}}

\def\atAllTimes{\pr{at-all-times}}
\def\atSomeTimes{\pr{at-some-times}}
\def\atAllTimesForWhichSubjectExists{\pr{at-all-times-for-which-subject-exists}}

\def\CellNucleus{\pr{cell nucleus}}
\def\Cell{\pr{cell}}

\def\OBOREL{\textbf{OBO-REL}}

\newcommand{\tbleqn}[1]{
\begin{math}
\begin{aligned}[1]
#1
\end{aligned}
\end{math}
}

\begin{document}
\firstpage{1}

\title{A critique of rigid temporalized relations}

\author{Christopher J. Mungall\,$^{1}$\footnote{to whom correspondence should be addressed}}
\address{$^{1}$Genomics Division, Lawrence Berkeley National Laboratory, MS84R017, 1 Cyclotron Road, Berkeley, CA 94720 USA}

\history{}

\editor{}

\maketitle

\begin{abstract}

  In this review I evaluate the proposed new temporalized relations
  strategy in which many existing relations would be replaced by two
  or more relations, an \emph{at-all-times}\ form and an
  \emph{at-some-times}\ form.

  My findings are that the \emph{at-all-times}\ relations have an
  underlying logical problem that renders them formally incorrect for
  use in many ontologies. The \emph{at-some-times} relations are
  safer, but would lose crucial transitive inferences. These logical
  problems are compounded by the fact that the relations are difficult
  for users and ontology developers to understand, and will most
  likely lead to confusion and errors, especially in the absence of
  detailed documentation.

  I conclude that these relations should not be adopted by ontology
  developers. Migrating to these relations would be an expensive,
  error-prone process that would alienate the user base of an
  ontology, and the end result would be ontologies that are either
  formally incorrect or too weak to perform required inferences.

\end{abstract}

\section{Introduction}

The OBO relations ontology (\OBOREL) defined a set of core relations
for use in biological ontologies, including \isA, \partOf\ and
\derivesFrom\cite{Smith2005}. The original \OBOREL\ paper has been
cited 709 times\footnote{Google scholar}, and has been a crucial
reference in the correct usage of relations in biological ontologies.

One notable aspect of \OBOREL\ was the precise specifications of how
relationships change or remain the same through the passage of
time. For example, the intent was that an ontology could state that
\emph{every} \CellNucleus\ is part of \emph{some} \Cell\ at \emph{any}
given moment of time (for which that cell nucleus exists). This has a
precise interpretation in first-order logic:

$
\A x \A t : \instanceOf(x, \CellNucleus, t) \imp \\
 \E y \instanceOf(y, \Cell, t), \partOf(x,y,t)
$

A relationship of this form is known as \emph{permanent generic
  parthood}.

The other notable aspect of \OBOREL\ was the distinction between type
(class) level and instance level relations. Each type-level relation
connects a pair of classes and typically is defined according to an
\pr{ALL-SOME-ALLTIMES}\ pattern - for example, every cell nucleus is
part of some cell at all times. The stated use of class level
relations causes some confusion when using OWL, which does not support
these kinds of class-level relationships. Instead the relationship
between a nucleus and a cell in OWL is explicitly quantified, but
without a time argument, as all relations in OWL are binary: For
example, \emph{every cell nucleus is part of some cell}. As of 2010,
the official semantics of OBO format have been as a subset of OWL, so
what applies to OWL necessarily applies to OBO.

This impedance mismatch between the OWL interpretation and the
\OBOREL\ account has been problematic in providing a consistent
formal account of relations that is consistent with OWL semantics,
although it is not clear that this has caused a problem for many
ontology developers or users. The standard approach has been to use a
set of binary relations as specified in Table
\ref{tab:characteristics-atemporal}, and to assume an \OBOREL\ type
interpretation for time.

As part of the release process for the OWL translation of version 2 of
the Basic Formal Ontology (BFO)\cite{Grenon2004}, a number of people
explored different strategies for unifying OWL binary properties with
the ternary relations in the BFO2 reference
specification\cite{Graz}. The goal was to find a way of representing
the temporal aspects of BFO2 relations in OWL in a formally satisfying
way.  One such strategy is the \emph{temporalized relations} (TR)
strategy, in which each reference relation relating continuants has
two or more OWL cognates, \pr{rel-at-some-times} and
\pr{rel-at-all-times}. This strategy has been adopted as the official
one for the OWL translations of BFO2.

In this review I do not attempt to compare or even describe the
different modeling possibilities; instead I focus purely on the
temporalized relations strategy, as there is pressure to make this the
standard for OBO library ontologies. I first provide an outline of
temporalized relations, drawing on the existing release notes and
documentation, attempting to fill some gaps. I then present the major
problems posed by these relations: (1) the relations fail to capture
the biological reality, forcing ontology editors to make a choice
between two unsatisfactory options, and ruling out the use of
non-rigid classes (2) the relations are confusing for both users and
experienced ontology editors. These two problems are related, in that
even experienced ontology editors may not understand the choices they
are being asked to make with TRs.

Finally I end with some specific recommendations regarding the
temporalized relation strategy in general and BFO2 specifically.

My intentions are to make this review accessible to a wide audience,
and to keep logical formulas to a minimum. This is difficult because
one of the problems with the TR strategy is that it forces complexity
upon the user and developer of an ontology, requiring some discussion
of that complexity in an attempt to explain the flaws.

\begin{table}
\begin{tabular}{ | p{2.5cm} | p{1cm} | p{1cm} | p{2.5cm} | }
\hline
\textbf{Relation} & \textbf{Trans} & \textbf{Symm} & \textbf{Inverse Of}  \\
\hline
\partOf\  & Yes & No & \hasPart  \\
\hline
\hasPart\ & Yes & No & \partOf \\
\hline
\adjacentTo\ & No & Yes &  \\
\hline
\end{tabular}
\caption{Relation characteristics, atemporal. These are the characteristics of the instance level relations in the current RO}
\label{tab:characteristics-atemporal}
\end{table}


\section{Temporalized Relations}

Here I distinguish between reference relations (RRs) and their
manifestation in OWL as binary temporalized relations (TRs), using the
temporalized relation strategy (TRS). All relations are instance
level. As a typographic convention I use dashes to separate the words
in a temporalized relation, and underscores in a reference relation.

\subsection{Translation template}

The BFO2 Graz version release notes\cite{Graz} specify a general
template for relating RRs to TRs:

\begin{verbatim}
x rel-at-some-time y ->
 exists(t) exists_at(x,t) -> 
    exists_at(y,t) and rel(x,y,t)
x rel-at-all-times y ->
  forall(t)  exists_at(x,t) -> 
      exists_at(y,t) and rel(x,y,t)
\end{verbatim}

Here we focus on \partOf\ as an exemplar relation, whilst recognizing
that similar patterns may apply to other, but not all relations.

For \partOf\ connecting two continuants there are in fact \emph{three}
TRs rather than two (for reasons that will be explained shortly):

\begin{enumerate}

\item \partOfAtSomeTimes

\item \partOfAtAllTimes

\item \partOfAtAllTimesForWhichWholeExists

\end{enumerate}

Note that in BFO2 the actual labels are ``part of continuant at some
time'', ``part of continuant at all times'' and ``part of continuant
at all times for which whole exists'', we shorten this for brevity --
here we are only concerned with relations that involve a continuant.

\subsection{TRs force a different interpretation from OBOREL}


On the surface, the \atAllTimes\ form appears to be the same as the
\OBOREL\ interpretation. For example, the following OWL axiom may
appear in an ontology that uses TRs:

$$
\CellNucleus\ \pr{SubClassOf}\ \partOfAtAllTimes\ \pr{some}\ \Cell
$$

This may seem to be the same as the statement ``every cell nucleus is part
of some cell at all times''. We might even be able to automatically
translate an ontology written using the \OBOREL\ interpretation into
TRs. \emph{However, these are NOT the same, and an understanding of
  why is crucial to correct usage of these relations and an
  understanding of the consequences of using them}.

It is not the case that TRs are the same as what has come before, but
with harder to read labels. \emph{The semantics are fundamentally
  different}. Whereas \OBOREL\ allowed permanent generic parthood (in
which a nucleus must always be part of a cell, but can be transferred
between cells), with TRs that possibility is disallowed. See Table
\ref{tab:fol-class-axioms} in the appendix for details.

The fact that these are different is of utmost importance to how
ontologies are created, and affects the \emph{characteristics} of
these relations in some ways that might seem surprising.

\subsection{Object property characteristics}

In OWL, relations (object properties) can have certain characteristics
such as being \emph{transitive}, \emph{symmetrical}, and they may be
related to other relations via logical axioms such as \emph{inverse
  properties} and property chains. These are extremely useful for many
purposes - transitivity has been at the core of bioinformatics
applications of ontologies from the initial version of the Gene
Ontology\cite{Ashburner2000}, if not before. Inverse properties are
useful for instance level reasoning, and for finding errors in complex
ontologies.

When translating a RR to a TR, it may not be immediately clear what
properties of the RR should be carried over to the TR. Ideally we
would be able to prove that the RR and TR are consistent, although the
proof may be obvious to an expert logician. Here I use \partOf\ as an
illustrative example, although each relation may require individual
examination for its properties.

Table \ref{tab:characteristics-temporalized} shows some of the
property characteristics of the various forms of the continuant
parthood relations. Figure \ref{fig:part} shows this in graphical
form.

%----------------------------------------
\begin{figure}
\center
\includegraphics[width=7cm]{part}
\caption{Parthood relations, both atemporal (A) and temporalized (B)}
\label{fig:part}
\end{figure}
%----------------------------------------


\begin{table}
\begin{tabular}{ | p{3cm} | p{1cm} | p{4cm} | }
\hline
\textbf{Relation} & \textbf{Trans} & \textbf{Inverse Of}  \\
\hline
\partOf\ \atSomeTimes & No & \hasPart\ \atSomeTimes \\
\hline
\partOf\ \atAllTimes & Yes & \hasPartAtAllTimesForWhichPartExists \\
\hline
\partOfAtAllTimesForWhichWholeExists & Yes & \hasPart\ \atAllTimes \\
\hline
\hasPart\ \atSomeTimes & No & \partOf\ \atSomeTimes \\
\hline
\hasPart\ \atAllTimes & Yes & \partOfAtAllTimesForWhichWholeExists \\
\hline
\hasPartAtAllTimesForWhichPartExists & Yes & \partOf\ \atAllTimes \\
\hline
\end{tabular}
\caption{Relation characteristics for core continuant parthood relations. These characteristics are declared in the current BFO2 OWL Graz version}
\label{tab:characteristics-temporalized}
\end{table}

\subsubsection{Transitivity}

In the case of the RR \partOf\ (which is transitive), the stronger
\atAllTimes\ TR retains the transitivity characteristic, whereas the
weaker \atSomeTimes\ TR does not have this. This means the weaker
version is often safer to use in an ontology, but will lead to fewer
inferences.

\subsubsection{Symmetricality}

For other relations and other characteristics, the translation may not
be obvious. For example, the adjacency relation is commonly assumed to
be symmetric\footnote{we are only considering instance level
  relations}. Should this symmetricality characteristic be carried
over to the temporalized form?

Currently BFO2 does not have an adjacency relation, or any other
reference relation that is symmetric, so the following is based on my
own understanding, and is shown in \ref{tab:characteristics-adj}. I would assume that the symmetricality should be
declared for the \atSomeTimes\ form and not the \atAllTimes\ form. In
contrast to the \partOf\ and \emph{transitivity}, here it is the
\emph{weaker} form of relation that inherits the characteristic.

\subsubsection{Properties}

We assume a hierarchy in which \atSomeTimes\ is the most general, with
\atAllTimesForWhichSubjectExists\ intermediate, and \atAllTimes\ most
specific. This is illustrated for parthood in figure \ref{fig:part}.

Note how this interacts with other properties. If:
$$x\ \adjacentTo\ \atAllTimes\ y$$

then we can infer that 

$$y\ \adjacentTo\ \atSomeTimes\ x$$.

\subsubsection{Inverse Properties}

Ontologies frequently declare inverse relations. For example, the
RR \partOf\ is the inverse of \hasPart. This is fairly standard
practice, and the inverse relations are extremely useful for
reasoning. In OWL, it's not strictly necessary to declare an inverse,
as it is possible to use an InversePropertyExpression. Here we take
the view that inverses that have typically been declared in previous
ontologies are useful, and should also be declared as RRs and have
corresponding TRs.

The inverse of the TRs of \partOf\ may not be completely
intuitive. Naively we might guess that \partOfAtAllTimes\ would be the
inverse of \hasPartAtAllTimes, but this not the case.

In fact, the declaring the inverse of \partOfAtAllTimes\ requires
declaring a third TR form, \hasPartAtAllTimesForWhichPartExists. The
\atAllTimesForWhichSubjectExists\ form can be generated for some but
not all RRs. Whilst it complicates the ontology of TRs to introduce
this extra form, the alternative of not having inverses (or of being
forced to write complicated inverse expressions) may be too
prohibitive for many users.




\subsubsection{Other characteristics}

We do not consider other characteristics such as anti-symmetry and
domain/range restrictions here.


\subsubsection{Generating a TR from an RR}

Currently there is no ``recipe'' for generating a set of TRs from an
RR. Different patterns may apply to different RRs.

For example, when creating the TRs for inheresIn and its inverse
bearerOf, it is correct to declare bearerOfAtAllTimes InverseOf
inheresInAtAllTimes.




\begin{table}
\begin{tabular}{ | p{3cm} | p{1cm} | p{4cm} | }
\hline
\textbf{Relation} & \textbf{Symm} & \textbf{Inverse Of}  \\
\hline
\adjacentTo\ \atSomeTimes & Yes & \adjacentTo\ \atSomeTimes \\
\hline
\adjacentTo\ \atAllTimes & No & \\
\hline
\end{tabular}
\caption{Relation characteristics for a typical symmetric relation. This relation is not part of the BFO2 OWL Graz version}
\label{tab:characteristics-adj}
\end{table}


\section{Evaluation}

\subsection{Temporalized Relations do not reflect the intentions of
  ontology editors}

TRs present many challenging problems. For example, when converting an
anatomy ontology that has been modeled traditionally using the
relations in table \ref{tab:characteristics-atemporal}, the ontology
editor must make a choice on a per-relationship basis as to which of
the relations in \ref{tab:characteristics-temporalized} should be
used.

This is an onerous task, but this could be justified if the results
were better ontologies. However, in many cases \emph{none of the
  choices are appropriate}.

This is because for many ontologies, the most appropriate choice of
parthood relationship is the \emph{permanent-generic} form, as
specified in the original OBO relations paper. The standard example
here is the relationship between a cell nucleus and a cell. At any
moment in time, a given cell nucleus is by definition part of some
cell\footnote{we would consider extruded nuclei to be transformations
  of cell nuclei, but instantiating a different class}. However,
\emph{this need not be the same call throughout the lifetime of the
  nucleus}.

In contrast, if the \partOfAtAllTimes\ TR is used then the
interpretation is that the cell nucleus is always part of the same
cell. This interpretation can be proved to be formally wrong in cells
that undergo cell division\ref{CellDiv}. This is because the cell
nucleus is always part of a cell, \emph{but not the same cell}.

The ontology editor can not then choose to use \partOfAtAllTimes\
without making a false statement. They may then decide to
use \partOfAtSomeTimes. Such a usage would be formally valid, but
incomplete from the point of view of useful reasoning. This is because
the weaker \atSomeTimes\ form lacks the transitivity characteristic.

A third possibility is to use
the \partOfAtAllTimesForWhichWholeExists\ form, but this would also be
false (note the above proof needs to be extended to cover this case).

The ontology editor requires the permanent generic form in order to be
both accurate and to get the required inferences. However, this form
is specifically excluded in the TR strategy.

The cell nucleus example is the standard one, because it is central to
all of eukaryotic biology. It is by no means the only such
example. The problem arises whenever we have material passed around
from one carrier to another.  Enumerating a list of examples is
difficult because the instance level identity conditions may not be
clear. 

the problem is not limited to parthood relations. Use of TRs requires
that all continuant relations are temporalized. This includes
relations used to classify structures by phenotype.

\subsection{Rigidity requirement is too onerous}

The BFO2 Graz release notes state:

\begin{quotation}

  Thus we only instantiate ``rigid'' classes, as the interpretation we
  take is a rdf:type C => forall(t) a exists at t -> a instance of C
  at t. Temporally restricted instantiation is not supported in this
  version of BFO in OWL. We are working on it for the future.

\end{quotation}

To many users this may seem like an obscure point, but it is actually
quite a sever restriction. A class is rigid if it is instantiated
``for life''. If an individual transforms from being an instance of
one class of thing to another, then those classes are not rigid.

An example of a rigid class may be ``Homo sapiens''. If an individual
instantiates this class at some time t, then they instantiate it all
times for which they exist (barring some unusual inter-species
transformation).

This clause means material entity classes such as the following may
not be supported in this version of BFO:

\begin{enumerate}

  \item 'human with Parkinson's disease'
  \item 'female organism'
  \item 'infected lung'
  \item 'professor' (but \emph{professor role} is allowed)
  \item 'human patient' (but \emph{patient role} is allowed)
  \item 'oocyte'
  \item 'fractured bone'
  \item 'happy human'
  \item 'fetal heart'
  \item 'neural crest cell'
  \item 'open heart valve'
  \item 'gravid uterus'
  \item 'phosphorylated EGFR protein'
  \item 'cytoplasmic NFkB'
  \item any leaf node from PATO

\end{enumerate}

In some cases the constraint may not be so onerous. It can be argued
that a well structured ontology would never include a class
``professor'', and that this should always be modeled using a rigid
class (human) plus a role (professor role). However, these decisions
should be made on a case by case basis by each ontology that than
imposed from the upper ontology.

In other cases the distinction between rigid and non-rigid may not be
clear. One can argue that when an EGFR protein changes state from
being unphosphorylated to a phosphorylated state it is no longer the
same instance - the protein literally ceases to exist and is replaced
by a distinct individual an instant later, sharing all the same
properties except that it is phosphorylated. In fact one could take
this position for all of the above cases, in which case the TR
strategy becomes similar to the temporally qualified continuant (TQC)
strategy. I do not explore this further, as I assume this is contrary
to the expectations of the TR proponents.

When considering BFO2 in particular, there is an interesting
disjunction between the reference document, which explicitly states
that \emph{determinates} (for example, qualities such as ``square'',
``charged'', ``cylindrical'') are non-rigid. The Graz release states
that instantiation of these classes is not supported. These two
seemingly contradictory statements are not explicitly linked
anywhere. The modeling implications of this disconnect are not clear,
and require further documentation. It cannot be ruled out that this
restriction will involve further complexity.

It may be the case that future versions of the TR strategy will allow
for non-rigid classes. It is not clear how this will be achieved
without additional complexity.  The TR strategy must be evaluated on
what exists presently, and at this time the strategy comes with
constraints that ontology developers should be fully aware of.

\subsection{Temporalized relations add complexity}

The most striking feature of an ontology that uses the TR strategy is
the complexity. Whereas using traditional modeling, we may have has a
single parthood relation, we now have three.

In theory some of this complexity could be tamed by additional tooling
(although it is not clear who has the resources to implement
this). However, even if this can be hidden from the user, the ontology
developer is forced to wrestle with the complexity.

This complexity first manifests when an ontology developer chooses to
migrate from a traditionally modeled ontology using relations from
table \ref{tab:characteristics-atemporal}, assuming an \OBOREL\
interpretation.

For each axiom that uses a relation that has multiple variants in TR
form, the developer must make a choice of which one to use....

BAD SMELL

\subsection{Case study: HDOT}




\section{Discussion}

\subsection{Recommendations}

\subsubsection{Do not use} My primary recommendation is that
Temporalized Relations should not be used in their current
form. Ontologies should not migrate to them.

\subsubsection{Documentation} The TR strategy needs much more
documentation if ontology developers are to use TRs. Even if TRs are
abandoned in their current form (as I recommend), documentation would
be useful to be able to help achieve consensus on this matter.

\subsubsection{Alternate strategies} Given the inherent limitations
and complexity of TRs, adequate consideration should be given to
alternate strategies such as Temporally Qualified Continuants
(TQCs). The ``null'' strategy of continuing to use simple OWL object
properties as if they has a \OBOREL\ interpretation should be the
default strategy until an adequate replacement is found.

\subsubsection{Use cases} If adopted, TRs will require tremendous
effort in ontology migration, documentation, tooling. There is little
to motivate ontology developers to do this as the existing strategy
works for them. The main motivating factor seems to be a desire for
formal correctness, at the expense of usability and biological
correctness.

\subsubsection{Road map}

\subsubsection{Smooth transition}

\section{Conclusions}

Temporalized Relations would be a fundamental change to the way
relationships are modeled in ontologies. They would introduce
additional complexity.

Some of these costs could be justified if Temporalized Relations were
on a path towards making ontologies more biologically
accurate. However, due to the built in lack of support for permanent
generic parthood and non-rigid classes, migrating to Temporalized
Relations would lead to ontologies becoming \emph{less} accurate. My
recommendation is unambiguous in its rejection of the use of
Temporalized Relations in biological ontologies.

\section*{Acknowledgments}



%\section*{References}

% ========================================
\bibliography{trc}
\bibliographystyle{plain}
% ========================================

\newpage
\section*{Appendix}

This section contains some additional material on the first order
logic axioms supporting the Temporalized Relations. 

SECTION NOT COMPLETE IGNORE FOR NOW

\begin{table}
\begin{tabular}{ | p{3cm} | p{4cm} | }
\hline
\textbf{Relation} & \textbf{Axiom}  \\
\hline
$$x\ \pr{part-of-}\ \atAllTimes\ y$$ &
        \tbleqn{
 & \dimp  \A t \existsAt(x,t) \imp\ \\
 & \existsAt(y,t), \partOf(x,y,t)
} \\
\hline
$$x\ \pr{part-of-}\ \atSomeTimes\ y$$ &
        \tbleqn{
 & \dimp  \E t \existsAt(x,t) \imp\ \\
 & \existsAt(y,t), \partOf(x,y,t)
} \\
\hline
\end{tabular}
\caption{Temporalized relations axioms for parthood relations. Taken from \cite{Graz} and transcribed into FOL syntax}
\label{tab:fol-part-temporalized}
\end{table}


\begin{table}
\begin{tabular}{ | p{1.8cm} | p{6.2cm} | }
\hline
\textbf{Axiom}  \\
\hline
\partOf (\OBOREL) &
        \tbleqn{
 & \A x \A t : \instanceOf(x, \CellNucleus, t) \imp \\
 & \E y : \instanceOf(y, \Cell, t), \partOf(x,y,t)
} \\
\hline
\pr{part-of-} \atAllTimes &
        \tbleqn{
 & \A x \E t : \instanceOf(x, \CellNucleus) \imp \\
 & \E y \instanceOf(y, \Cell), \A t \existsAt(x,t) \imp\ \\
 & \existsAt(y,t), \partOf(x,y,t)
} \\
\hline
\pr{part-of-} \atSomeTimes &
        \tbleqn{
 & \A x \E t : \instanceOf(x, \CellNucleus) \imp \\
 & \E y \instanceOf(y, \Cell), \E t \existsAt(x,t) \imp\ \\
 & \existsAt(y,t), \partOf(x,y,t)
} \\
\hline
\end{tabular}
\caption{
  Semantics of class axioms with parthood example. The first row shows the
  biologically correct relationship (permanent generic parthood), given by \OBOREL\ semantics. The next two rows show 
  two of the temporalized options - neither of these is equivalent to the \OBOREL\ version. 
}
\label{tab:fol-class-axioms}
\end{table}


\end{document}
