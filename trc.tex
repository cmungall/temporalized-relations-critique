\documentclass{bioinfo}
%\documentclass{article}
%%%%%%%%%%%%%%%%%%%%%%%%%%%%%%%%%%%%%%%%%%%%%%%%%%%%%%%%%%%%%%%
%%                    ** mymacros.tex **                     %%
%%%%%%%%%%%%%%%%%%%%%%%%%%%%%%%%%%%%%%%%%%%%%%%%%%%%%%%%%%%%%%%


\typeout{** loading mymacros.tex **}

\def\h{\hbox}

\newcommand{\mathcom}[3]{ \newcommand{#1}[#2]{\mbox{$#3$}}}
\newcommand{\remathcom}[3]{ \renewcommand{#1}[#2]{\mbox{$#3$}}}

\newcommand{\mdef}[2]{ \newcommand{#1}{\mbox{$#2$}} }
 
\mathcom{\y}{0}{\ \vdash\ }               % yeilds 
\mathcom{\ys}{0}{\vdash_{\cal S}}         % yeilds sub S
\mathcom{\yt}{0}{\vdash_{\theta}}         % yeilds sub theta

\mathcom{\absurd}{0}{\mathbf{f}}                 % absurdity
\mathcom{\imp}{0}{\ \rightarrow\ }            % implication arrow
\mathcom{\rimp}{0}{\ \leftarrow\ }            % implication arrow

\mathcom{\con}{0}{\ \wedge\ }                 % conjunction
\mathcom{\dis}{0}{\ \vee\ }                   % disjunction
\mathcom{\n}{0}{\neg}                     % negation
\mathcom{\dimp}{0}{\ \leftrightarrow\ }       % mat equiv

\mathcom{\corresponds}{0}{\ \Lleftarrow\! \! \Rrightarrow\ }


%%\mathcom{\th}{0}{\theta}                  % theta


\mathcom{\A}{0}{\forall}                  % universal quantifier
\mathcom{\E}{0}{\exists}                  % existential quant

\mathcom{\Dec}{1}{{\cal D}(#1)}           % D(#1)

\mathcom{\elt}{0}{\in}

\mathcom{\tuple}{1}{\langle #1 \rangle}

% \mathcom{\equivdef}{0}{\equiv_{\mbox{\em \tiny def}}}
% \mathcom{\eqdef}{0}{=_{\mbox{\em \tiny def}}}
\def\equivdef{\mathrel{\ \equiv_{\mbox{\em \tiny
def}}\ }}

\def\eqdef{\mathrel{\ =_{\mbox{\em \tiny def}}\ }}


% \newtheorem{Rule}{Rule}
% \newtheorem{Lemma}{Lemma}
% \newtheorem{Theorem}{Theorem}


% Redeclaration of symbols for intuitionistic logic
% see Diller p.126
\def\ineg{\mathop\sim}
\def\iimp{\mathbin\Rightarrow}
\def\iiff{\mathbin\Leftrightarrow}
% new symbols from amssymb
%\def\iimp{\mathbin\rightarrowtail}
%\def\iiff{\mathbin\leftrightsquigarrow}

\def\icon{\mathbin\curlywedge}
\def\idis{\mathbin\curlyvee}

%Definitions for RCC inserts
\newcommand{\notch}{\Longrightarrow\kern-16pt{/}\ \ \ }
\newcommand{\ch}{\Longrightarrow}
\newcommand{\pr}[1]{\mbox{\sf #1}}

% RCC relations
\mathcom{\C}{0}{\pr{C}}
\remathcom{\P}{0}{\pr{P}}
\remathcom{\O}{0}{\pr{O}}
\mathcom{\PO}{0}{\pr{PO}}
\mathcom{\PP}{0}{\pr{PP}}
\mathcom{\NTPP}{0}{\pr{NTPP}}
\mathcom{\NTPPI}{0}{\pr{NTPP}^{-1}}
\mathcom{\TPP}{0}{\pr{TPP}}
\mathcom{\TPPI}{0}{\pr{TPP}^{-1}}
\mathcom{\TP}{0}{\pr{TP}}
\mathcom{\NTP}{0}{\pr{NTP}}
\mathcom{\NTPi}{0}{\pr{NTPi}}
\mathcom{\EC}{0}{\pr{EC}}
\mathcom{\DC}{0}{\pr{DC}}
\mathcom{\DR}{0}{\pr{DR}}
\mathcom{\EQ}{0}{\pr{EQ}}
\mathcom{\CG}{0}{\pr{CG}}

\mathcom{\Pin}{0}{\pr{Pi}} % \Pi is Greek letter
\mathcom{\PPi}{0}{\pr{PPi}}
\mathcom{\NTPPi}{0}{\pr{NTPPi}}

\mathcom{\OP}{0}{\pr{OP}}

\mathcom{\conv}{0}{\pr{conv}}
\mathcom{\rsum}{0}{\pr{sum}}
\mathcom{\rdiff}{0}{\pr{diff}}
\mathcom{\rcompl}{0}{\pr{compl}}
\mathcom{\rprod}{0}{\pr{prod}}
\mathcom{\cp}{0}{\pr{cp}}
\mathcom{\Us}{0}{\pr{Us}}
\mathcom{\us}{0}{\pr{u}}
\mathcom{\NULL}{0}{\pr{NULL}}
\mathcom{\rnull}{0}{\emptyset}
\mathcom{\CONV}{0}{\pr{CONV}}
\mathcom{\CON}{0}{\pr{CON}}

% Adaptations for new latex compatibility
\renewcommand{\Box}{\square}
\def\cal{\mathcal}

% meta language in definitions etc
\mathcom{\ml}{1}{\;\;\;\; \mbox{#1} \;\;\;\;}


%-- TOPLOG SYMBOLS

\mathcom{\Co}{0}{{\cal C}_{0} \ }
\mathcom{\CoX}{0}{{\cal C}_{0}}
\mathcom{\Cop}{0}{{\cal C}_{0}^+ \ }
\mathcom{\CopX}{0}{{\cal C}_{0}^+}
\mathcom{\mcop}{0}{\models_{C_o^+}}

\mathcom{\Lop}{0}{{\cal L}_{0}^+ \ }
\mathcom{\LopX}{0}{{\cal L}_{0}^+}
\def\mlo{\models_{{\mathcal  L}_0}}
\def\mlop{\models_{{\mathcal  L}_0^+}}

\mathcom{\Ci}{0}{{\cal C}_{1} \ }
\mathcom{\CiX}{0}{{\cal C}_{1}}

\mathcom{\Io}{0}{{\cal I}_{0} \ }
\mathcom{\IoX}{0}{{\cal I}_{0}}
\mathcom{\Iop}{0}{{\cal I}_{0}^+ \ }
\mathcom{\IopX}{0}{{\cal I}_{0}^+}

%%\mathcom{\inter}{1}{\langle #1 \rangle}
\mathcom{\inter}{1}{i( #1 )}

\newfont{\myssfont}{cmss12}

\mathcom{\Maps}{2}{\, _{#1}\!\!\!\rightleftharpoons^{#2} \ }

\mathcom{\CotoST}{0}{\Maps{C_0}{ST}}
\mathcom{\CtoST}{0}{\Maps{C\,}{ST}}
\mathcom{\IotoST}{0}{\Maps{I_0}{ST}}

\mathcom{\uni}{0}{\cal U \, }
\mathcom{\uniX}{0}{\cal U}
\mathcom{\U}{0}{\cal U}

\mathcom{\eu}{0}{\,=\,\uni}

\mathcom{\ol}{1}{\overline{#1}}

\mathcom{\Lsse}{0}{{\cal L}_{sse}}
\mathcom{\Lssei}{0}{{\cal L}_{ssei}}

\mathcom{\Luse}{0}{{\cal L}_{use}}
\mathcom{\Lusei}{0}{{\cal L}_{usei}}

\mathcom{\disdots}{0}{\dis\!\!\!\ldots\!\!\dis}
\mathcom{\condots}{0}{\con\ldots\con}





%% Miscelleneous


\def\HA#1#2{\pr{Holds-At}(#1,#2)}
\def\col{\!:\!}

% Enclose arguments in square double brackets to
% refer to semantic value of an expression
\def\semvalue#1{[\mkern-3mu[#1]\mkern-3mu]}

\def\forces{\mathop{\setbox0=\hbox{$\vdash$}\ \rule{0.04em}{1\ht0}\mkern1.5mu\box0}}

\mathcom{\hence}{0}{\Longrightarrow \hspace{2em}}

\def\sliderule{\centerline{\rule{4in}{1mm}}}

\def\Cbox{\mathop{\square\mkern-10mu
                   \mbox{\raise0.45ex\hbox{\tiny\sf C}}
                  \mkern2mu
                 }
         }

\def\letterbox#1{\mathop{\square\mkern-10mu
                   \mbox{\raise0.2ex\hbox{\small\sf #1}}
                  \mkern2mu
                 }
         }

%\def\tBox{\letterbox{\lower0.4ex\hbox{t}}}
\def\sBox{\letterbox{s}}
\def\tBox{\letterbox{t}}

\def\overstrike#1#2{\mathop{%
               \setbox1=\hbox{$#1$}%
               \setbox2=\hbox{$#2$}%
               \ifdim 1\wd1<1\wd2 % 
                    { \rlap{\mbox{\kern0.5\wd2\kern-0.5\wd1 \box1}} \box2 }
              \else {   \rlap{\mbox{$#1$}}
                        \rlap{\mbox{\kern0.5\wd1\kern-0.5\wd2 \box2}}
                        \phantom{#1} } 
              \fi }}

\def\rbox{\boxminus}
\def\rdia{\overstrike{\Diamond}{-}}
\def\cdia{\overstrike{\Diamond}{\rule{0.04em}{1.5ex}}}
\def\cbox{\inbox{\rule{0.04em}{1.6ex}}}
\def\cbox{\mathop{\setbox0=\hbox{$\square$}
                  \overstrike{\square}{\hbox{\rule{0.04em}{1\ht0}}}
             }}

\def\diaplus{\overstrike{\Diamond}{+}}




%% Make a modal box with a symbol centred inside
\def\inbox#1{\mathop{\setbox0=\hbox{$\square$}%
                      \setbox1=\hbox{#1}%
                      \rlap{\hbox{$\square$}}
                      \rlap{\mbox{\kern0.5\wd0\kern-0.5\wd1%
                                  \box1}}%
                      \phantom{\square}}}

\def\ibox{\mathop{\square\mkern-10mu
                   \mbox{\raise0.45ex\hbox{\tiny\em i}}
                  \mkern7mu
                 }
         }
\def\cdiamond{\mathop{\Diamond\mkern-12.7mu
                   \mbox{\raise0.4ex\hbox{\tiny\em c}}
                  \mkern7mu
                 }
         }


\def\s5box{\mathop{\square\mkern-10mu
                   \mbox{\raise0.45ex\hbox{\tiny 5}}
                  \mkern7mu
                 }
         }

\def\Fbox{\mathop{\square\mkern-11mu
                   \mbox{\raise0.45ex\hbox{\tiny \bf F}}
                  \mkern2mu
                 }
         }

\def\Box{\mathop\square}
\def\Diamond{\mathop\lozenge}

\def\bigI{\mathop{%
\hbox{%
%%
\def\thickness{1pt}%
\def\width{6pt}%
\def\height{0.6em}%
\def\gap{0.15em}%
%%
\dimen0=\thickness%
\divide\dimen0 by 2%
\dimen1=\width%
\divide\dimen1 by 2%
%%
\kern\gap%
\rlap{\vrule width\width height\thickness depth0pt}%
\rlap{\kern\dimen1 \kern-\dimen0%
\vrule width\thickness height\height depth0pt}%
\raise\height\hbox{\vrule width\width height\thickness depth0pt}%
\kern\gap}}}

\def\mconv{\mathop{% 
      \mbox{\raise0.35ex\hbox{$\scriptstyle\bigcirc$}}
                  }
          }
\def\mpconv{\circleddash}



% \def\putat#1#2#3{\setbox1=\hbox{#3}%
% \rule{0pt}{0pt}%
% \vspace*{#2}%
% \hspace*{#1}%
% \rule{0pt}{0pt}%
% \hbox to 0em{\rlap{\vtop to 0ex{\hbox to 0em{#3}}}}%
% \hspace*{-#1}\vspace*{-#2}}
% %\rule{0pt}{0pt}
% %\vspace*{-\ht1}\rule{0pt}{0pt}
% %\vspace*{-\ht1}

\def\putat#1#2#3{\rlap{\hbox{%
\kern#1% 
\vtop{\hbox{}%
\hbox{\vbox to #2{}} \hbox{#3}}}}}

% use putat within \hbox
% Objects will be located relative to baseline
% at start of the \hbox
% eg:
% \hbox{
% \putat{2in}{2in}{xxxxx}%
% \putat{2in}{2in}{00000}%
% \putat{5in}{3in}{Z}%
% \putat{2in}{2in}{0}%
% \putat{5in}{4in}{x}%
% }


%\long\def\putat#1#2#3{




\def\ri{\mathclose{\raise1ex\hbox{$\smile$}}}

\def\eqtag#1{\eqno ({\bf #1})}

\def\mc:#1{\hbox{${\mathcal{#1}}$}}
\def\mb#1{{\mathbf{#1}}}
 
\def\ifff{\qquad \hbox{if and only if} \qquad}

%%\def\d{\delta}

\def\QED{$\blacksquare$}

\def\epsrcgrant{GR/K65041}
\def\VUGgrant{GR/M56807}

\def\Boxx{\boxtimes}

%% Force a math symbol to go into scriptsize
\def\mscriptsize#1{\hbox{\scriptsize $#1$}}
\def\mlarge#1{\hbox{\large $#1$}}

\def\boxtheorem#1#2{~\\
\centerline{\fbox{
\parbox{5in}{
\centerline{\bf #1}
#2
}}}
\vspace{1ex}}

\def\proof#1#2{
\begin{quotation}
\noindent {\bf Proof of #1:}
#2
\QED
\end{quotation}
}

\def\proofpar{\\ \hspace*{1.5em}}


%%%% ENVIRONMENTS

%% specify my list parameters
%% These can be overridden by declarations within
%% a document.

\def\mytopsep{1ex}
\def\mypartopsep{0ex}
\def\myitemsep{0.3ex}
\def\myparsep{0ex}
\def\mylistparindent{2em}
\def\myleftmargin{4.5em}
\def\mylabelwidth{4.5em}
\def\mylabelsep{1em}
\def\myitemindent{0em}

\def\setmylistparams{%
             \setlength{\topsep}{\mytopsep}
             \setlength{\partopsep}{\mypartopsep}
             \setlength{\itemsep}{\myitemsep}
             \setlength{\parsep}{\myparsep}
             \setlength{\listparindent}{\mylistparindent}
             \setlength{\leftmargin}{\myleftmargin}
             \setlength{\labelwidth}{\mylabelwidth}
             \setlength{\labelsep}{\mylabelsep}
             \setlength{\itemindent}{\myitemindent}
        }

\def\clistbullet{$\bullet$}

\newcounter{tempvalue}
% A compact list making environment
\newenvironment{clist}
    { %\vspace{1ex}
      \begin{list}
            {\labelitemi}
            \setmylistparams
    }
    { \end{list}\addvspace{1ex} 
    }
% A compact list making environment
% with -- as the default item tag
\newenvironment{cdashlist}
    { %\vspace{1ex}
      \begin{list}
            {--}
            {\setlength{\topsep}{0.2ex}
             \setlength{\partopsep}{0ex}
             \setlength{\itemsep}{0.3ex}
             \setlength{\parsep}{0ex}
             \setlength{\listparindent}{2em}
            }
    }
    { \end{list}\addvspace{1ex} 
    }


% compact list with wide labels for item tags
\newenvironment{taglist}[1]
    { %\vspace{1ex}
      \begin{list}
            {$\bullet$}
            {\setlength{\topsep}{0.2ex}
             \setlength{\partopsep}{0ex}
             \setlength{\itemsep}{0.3ex}
             \setlength{\parsep}{0ex}
             \setlength{\listparindent}{2em}
             \setlength{\leftmargin}{#1}
             \setlength{\labelwidth}{#1}
             %\addtolength{\labelwidth}{-1em}
             \setlength{\labelsep}{0em}
            }
    }
    { \end{list}\addvspace{1ex} 
    }

\newcounter{cenumcount}
\newenvironment{cenum}
    { %\vspace{1ex}
      \begin{list}
            {\arabic{cenumcount}.}
            {\usecounter{cenumcount}
               \setmylistparams
            }
    }
    { \end{list}\addvspace{1ex} 
    }

\newcounter{romcount}
\newenvironment{romanenum}
          { 
            \begin{list}{\roman{romcount})~~}
                        {\usecounter{romcount}
                         \setmylistparams
                        }
          }
          { \end{list} }

\newcounter{alphcount}
\newenvironment{alphenum}
          { \begin{list}{\alph{alphcount})~~}
                        {\usecounter{alphcount}}}
          { \end{list} }


%% Labelled list environment
\newenvironment{llist}[1]{%
\newcounter{#1}
\begin{list}{({\bf #1\arabic{#1}})\hspace{1em}}{\usecounter{#1}}
}
{\end{list}}

%% Labelled list continued
%% (assumes counter already exists)
\newenvironment{llistcont}[1]{%
\setcounter{tempvalue}{\value{#1}}
\begin{list}{({\bf #1\arabic{#1}})\hspace{1em}}{%
\usecounter{#1}
\setcounter{#1}{\value{tempvalue}}
}
}
{\end{list}}

%%% compact versions of llist and llistcont
\newenvironment{cllist}[1]{%
\newcounter{#1}
\begin{list}{({\bf #1\arabic{#1}})}
             { \usecounter{#1}
               \setmylistparams
             }
}
{\end{list}}

%% Labelled list continued
%% (assumes counter already exists)
\newenvironment{cllistcont}[1]{%
\setcounter{tempvalue}{\value{#1}}
\begin{list}{({\bf #1\arabic{#1}})}{%
                 \usecounter{#1}
                 \setcounter{#1}{\value{tempvalue}}
                \setmylistparams
          }
}
{\end{list}}


\def\longdef{\long\def}

%% Define a labelled list type
%% #1 name of control sequence for the list type
%% #2 prefix for numbering
\def\llisttype#1#2{%
\newcounter{#2}
\expandafter\longdef\csname #1list\endcsname##1{%
\begin{cllistcont}{#2}
##1
\end{cllistcont}}
\expandafter\def\csname #1\endcsname##1{%
\begin{cllistcont}{#2}
\item
##1
\end{cllistcont}}
}

\def\litem#1#2{%
\begin{clist}
\item[({\bf #1})] 
#2
\end{clist}}

\newenvironment{chapabst}%
{\begin{quotation}\small\noindent }{\end{quotation}}

% Split line displayed formulae
% (not actually an environment)
\newcommand{\splitdismath}[4]{%
{\setlength{\arraycolsep}{0em}
\begin{eqnarray}
& #1 & #2 \nonumber \\
 &    & #3 
\label{#4}
\end{eqnarray}
}
}

% Displayed equation with tabbing

\newcommand{\tabmathn}[2]
{ \begin{equation}
\vbox{
\vspace{-2ex}
\begin{tabbing}
#1
\end{tabbing}
\vspace{-3.5ex}
}
\label{#2}
\end{equation}
}

\newcommand{\displaytab}[1]
{ \begin{displaymath}
\vbox{
\vspace{-2ex}
\begin{tabbing}
#1
\end{tabbing}
\vspace{-3.5ex}
}
\end{displaymath}
}

\def\myeq#1{$$\eqalign{#1}$$}


%%% Reinstate \eqalign macros removed by Lamport
\catcode`\@=11
\def\eqalign#1{\null \,\vcenter {\openup \jot \m@th 
\ialign {\strut \hfil $\displaystyle{##}$&$\displaystyle {{}##}$\hfil
 \crcr #1\crcr }}\,}

\def\eqalignno#1{\displ@y \tabskip \centering 
\halign to\displaywidth {\hfil $\@lign \displaystyle{##}$\tabskip \z@skip
 &$\@lign \displaystyle {{}##}$\hfil \tabskip \centering 
 &\llap {$\@lign ##$}\tabskip \z@skip \crcr 
 #1\crcr }}

\def\leqalignno#1{\displ@y \tabskip \centering
 \halign to\displaywidth {\hfil $\@lign \displaystyle {##}$\tabskip \z@skip
 &$\@lign \displaystyle {{}##}$\hfil \tabskip \centering
 &\kern -\displaywidth \rlap {$\@lign ##$}\tabskip \displaywidth \crcr 
 #1\crcr}}

%%% Define something like \bibliography but only records the
%%% bibfiles (does not input the bbl file)
\def\bibliographyfiles#1{%
  \if@filesw
    \immediate\write\@auxout{\string\bibdata{#1}}%
  \fi}

%%% Define a separate command to input the bbl file.
\def\bibliographytext{%
  \@input@{\jobname.bbl}}


%% Suppress printing of bibitems in thebibliography
%% This is for use with the `inlinebib' package,
%% which adds fullcitations where you give the \cite command.
%% Doesn't print note
%% Could add note by changing ##2 to ##2##3 in the \bibcite argument
%% But output not quite right.
\def\suppressendbib{%
\long\def\@bibitem##1 ##2 \note ##3 \short ##4 \end{\par\if@filesw
        {\def\protect####1{\string ####1\space}%
         \let\newblock\@empty
         \immediate\write\@auxout{\string
                \bibcite{##1}{\string\@bibcall{##1}{##2}{##4}}}}\fi
        }
\def\thebibliography##1{}
}


%% Change indent space allocated by \thebibliography
%% use before \bibliography or \bibliographytext 
\def\setbibindent#1{%
\let\oldthebibliography\thebibliography
\def\thebibliography##1{\oldthebibliography{#1}}
}



\catcode`\@=12

\long\def\skipover#1{}

\long\def\correction#1{\footnote{TO CORRECT: #1}}
\def\nocorrections{\long\def\correction##1{}}
\def\question#1{{\bf Question:} {\em #1} }
\long\def\todo#1{{\bf To Do:} {\em #1} }


\def\twiddle{$\sim$}

%% Macro for URLs
%% puts them in tt format and  breaks line after dot if needed.

\skipover{
\def\urlspace{-0.5em}
\def\url#1{{\tt \urlsub#1\end}}
%\def\urldot{. \hspace{-1em}\urlsub}
\def\urldot{. \linebreak[1]\hspace\urlspace\urlsub}
\def\urlcolon{: \linebreak[1]\hspace\urlspace\hspace\urlspace\urlsub}
\def\urlslash{/ \linebreak[1]\hspace\urlspace\urlsub}
\def\urltilde{\twiddle\urlsub}
\def\urlsub#1{\ifx#1\end \let\next=\relax
      \else \ifx#1.\let\next=\urldot
      \else \ifx#1:\let\next=\urlcolon
      \else \ifx#1/\let\next=\urlslash
      \else \ifx#1~\let\next=\urltilde
      \else#1\let\next=\urlsub\fi\fi\fi\fi\fi\next}
}

% Input a file in verbatim mode (eg a program file)
\def\verbinput#1{\expandafter\begin{verbatim}\input#1}
%NB: usage: \verbinput{filename}\end{verbatim}


\def\mb#1{{\mathbf{#1}}}
\def\mbf#1{{\mathbf{#1}}}


%%% MACROS for SLIDES


\def\heading#1{%
\centerline{\shadowbox{
\large \begin{tabular}{c} #1 \end{tabular}
           }  }
\vspace{1ex minus 1ex}
}

\def\bheading#1{%
\centerline{\shadowbox{
\large\bf \begin{tabular}{c} #1 \end{tabular}
           }  }
\vspace{1ex minus 1ex}
}

\def\printlandscape{\special{landscape}}    % Works with dvips.
\def\printportrait{\special{portrait}}

\def\leedslogo#1{\centerline{
\psfig{file=uol.eps,width=#1} %% May look wrong in xdvi
}}

\def\titleslide#1#2{
\begin{slide*}
~

\vfill
\heading{#1}

\vfill
\begin{center}
        {\large\bf Brandon Bennett} 

\vspace{3ex}
        Division of AI \\
        School of Computing \\
        University of Leeds\\ 
        Leeds LS2 9JT, England \\
        {\tt brandon@comp.leeds.ac.uk} \\
\end{center}

\vfill
\centerline{
\psfig{file=uol.eps,width=0.7in} %% May look wrong in xdvi
}
\vfill
\centerline{#2}
\vfill
\end{slide*}
}


\def\nlf{\\ \mbox{} \hfill}

%%\def\fixhyphens{\usepackage[english]{babel}}

%% FIX HYPHENATION
\lefthyphenmin=2
\righthyphenmin=3

%% special commands for spell checker
\def\sic#1{#1}
\def\sico!#1!{#1}
\def\sicname#1{#1}
\def\accept#1{}
\def\acceptname#1{}

%\newenvironment{nospell}{}{}
\def\spelloff{}
\def\spellon{}


%% Check whether a macro is defined
\def\ifundefined#1{\expandafter\ifx\csname#1\endcsname\relax}

%% Define \ifpdf
%% Used in graphic stuff to check if pdf is running
%% This is taken from ifpdf.sty
\newif\ifpdf
\ifx\pdfoutput\undefined
\else
  \ifx\pdfoutput\relax
  \else
    \ifcase\pdfoutput
    \else
      \pdftrue
    \fi
  \fi
\fi

\ifpdf
  \typeout{Running in PDF mode}
\else
  \typeout{Running in DVI mode}
\fi

%%%% CLEVER GRAPHIC INCLUSION

\def\pdfgraphictype{jpg} %% or perhaps pdf or png
\ifpdf   
  \def\graftype{\pdfgraphictype}  
\else
  \def\graftype{eps}
\fi


\typeout{* BB mymacros: Default graphic type set to: \graftype}

\ifpdf
   \def\optpdftex{pdftex}
   \def\optgraphic[#1]#2#3{\includegraphics[#1]{#3}} 
\else
   \def\pdfinfo#1{}
   \def\optpdftex{}
   \def\optgraphic[#1]#2#3{\includegraphics[#1]{#2}} 
\fi

\def\altgraphic[#1]#2#3#4{\optgraphic[#1]{#2.#3}{#2.#4}}

\def\figurepath{}
\def\graphswap[#1]#2{\optgraphic[#1]{\figurepath#2.eps}{\figurepath#2.\pdfgraphictype}}

%% \graphicifpdf includes graphic in pdf mode
%% otherwise just draws a box with the file name
%% I haven't really used this much.
\ifpdf
     \def\graphicifpdf[#1]#2{\includegraphics[#1]{#2}}
   \else
     \def\graphicifpdf[#1]#2{\fbox{\parbox{1in}{#1\\ #2}}}
\fi 

%\def\setaltgraphic#1#2{%
%   \def\altgraphic[##1]##2{\optgraphic[##1]{##2.#1}{##2.#1}}}

\def\graphic[#1]#2{\includegraphics[#1]{#2.\graftype}} 

\def\psfiggraphic{\renewcommand{\graphic}[2][]{\psfig{##1,file=##2.eps}}}

%%% Date functions

\def\dayth{\number\day\ifcase\day \or
st\or nd\or rd\or th\or th\or th\or th\or th\or th\or th\or 
th\or th\or th\or th\or th\or th\or th\or th\or th\or th\or
st\or nd\or rd\or th\or th\or th\or th\or th\or th\or th\or
st\fi}

\def\monthname{\ifcase\month \or
January\or February\or March\or April\or May\or June\or
July\or August\or September\or October\or November\or December\fi}


%%%% How to set up new font sizes
%% define a font which is Helvetica at 4.5 pt
%\newfont{\boxfont}{helvetica at 4.5 pt}

% How to rename or alter the References section
% \renewcommand\refname{References\footnote{The   reference  list  given
%     here is somewhat incomplete due to limited preparation time. It is
%     intended  to  add several  more  recent  references  to the  final
%     version of this paper, relating the discussion to current works on
%     ontology      development.}}      


\typeout{** finished loading mymacros.tex **}

%%%%%%%%%%%%%%%%%%%%%%%%%%%%%%%%%%%%%%%%%%%%%%%%%%%%%%%%%%%%%%%
%%  END END END         (of mymacros.tex)      END END END   %%
%%%%%%%%%%%%%%%%%%%%%%%%%%%%%%%%%%%%%%%%%%%%%%%%%%%%%%%%%%%%%%%

\usepackage{url}


\ifx\pdfoutput\undefined
% we are running LaTeX, not pdflatex
\usepackage{graphicx}
\else
% we are running pdflatex, so convert .eps files to .pdf
%\usepackage[pdftex]{graphicx}
%\usepackage{epstopdf}
\fi 

\copyrightyear{}
\pubyear{}

\def\partOf{\pr{part\_of}}
\def\hasPart{\pr{has\_part}}
\def\isA{\pr{is\_a}}
\def\instanceOf{\pr{instance\_of}}
\def\derivesFrom{\pr{derives\_from}}
\def\adjacentTo{\pr{adjacent\_to}}

\def\existsAt{\pr{exists\_at}}

\def\partOfAtSomeTimes{\pr{part-of-at-some-times}}
\def\partOfAtAllTimes{\pr{part-of-at-all-times}}
\def\hasPartAtSomeTimes{\pr{has-part-at-some-times}}
\def\hasPartAtAllTimes{\pr{has-part-at-all-times}}
\def\hasPartAtAllTimesForWhichPartExists{\pr{has-part-} \pr{at-all-times-}\ \pr{for-which-part-exists}}
\def\partOfAtAllTimesForWhichWholeExists{\pr{part-of-} \pr{at-all-times-}\ \pr{for-which-whole-exists}}

\def\atAllTimes{\pr{at-all-times}}
\def\atSomeTimes{\pr{at-some-times}}
\def\atAllTimesForWhichSubjectExists{\pr{at-all-times-for-which-subject-exists}}

\def\CellNucleus{\pr{cell nucleus}}
\def\Cell{\pr{cell}}

\def\OBOREL{\textbf{OBO-REL}}

\newcommand{\tbleqn}[1]{
\begin{math}
\begin{aligned}[1]
#1
\end{aligned}
\end{math}
}

\begin{document}
\firstpage{1}

\title{A critique of rigid temporalized relations}

\author{Christopher J. Mungall\,$^{1}$\footnote{to whom correspondence should be addressed}}
\address{$^{1}$Genomics Division, Lawrence Berkeley National Laboratory, MS84R017, 1 Cyclotron Road, Berkeley, CA 94720 USA}

\history{}

\editor{}

\maketitle

\begin{abstract}

  In this review I evaluate the proposed new temporalized relations
  strategy in which many existing relations would be replaced by two
  or more relations, an \emph{at-all-times}\ form and an
  \emph{at-some-times}\ form.

  My findings are that the \emph{at-all-times}\ relations have an
  underlying logical problem that renders them formally incorrect for
  use in many ontologies. The \emph{at-some-times} relations are
  safer, but would lose crucial transitive inferences. These logical
  problems are compounded by the fact that the relations are difficult
  for users and ontology developers to understand, and will most
  likely lead to confusion and errors, especially in the absence of
  detailed documentation.

  I conclude that these relations should not be adopted by ontology
  developers as a replacement for existing relations. Migrating to
  these relations would be an expensive, error-prone process that
  would alienate the user base of an ontology, and the end result
  would be ontologies that are either formally incorrect or too weak
  to perform required inferences.

\end{abstract}

\section{Introduction}

\subsection{The OBO Relations ontology}


The OBO relations ontology (\OBOREL) defined a set of core relations
for use in biological ontologies, including \isA, \partOf\ and
\derivesFrom\cite{Smith2005}. The original \OBOREL\ paper has been
cited 709 times\footnote{Google scholar}, and has been a crucial
reference in the correct usage of relations in biological ontologies.

One notable aspect of \OBOREL\ was the precise specifications of how
relationships change or remain the same through the passage of
time. For example, the intent was that an ontology could state that
\emph{every} \CellNucleus\ is part of \emph{some} \Cell\ at \emph{any}
given moment of time (for which that cell nucleus exists). This has a
precise interpretation in first-order logic:

$
\A x \A t : \instanceOf(x, \CellNucleus, t) \imp \\
 \E y \instanceOf(y, \Cell, t), \partOf(x,y,t)
$

A relationship of this form is known as \emph{permanent generic
  parthood}. It allows for nuclei to be passed from one cell to
another, so long as the nucleus remains part of some cell.

The other notable aspect of \OBOREL\ was the distinction between type
(class) level and instance level relations. Each type-level relation
connects a pair of classes and typically is defined according to an
\pr{ALL-SOME-ALLTIMES}\ pattern - for example, every cell nucleus is
part of some cell at all times. This causes some confusion when using
OWL, which does not support these kinds of class-level
relationships. Instead the relationship between a nucleus and a cell
in OWL is explicitly quantified, but without a time argument, as all
relations in OWL are binary: For example:

\begin{verbatim}
'cell nucleus' SubClassOf part_of some cell
\end{verbatim}

As of 2010, the official semantics of OBO format
have been as a subset of OWL, so what applies to OWL necessarily
applies to OBO.

This impedance mismatch between the OWL interpretation (which can be
considered ``atemporal'') and the \OBOREL\ account has been
problematic in providing a consistent formal account of relations that
is consistent with OWL semantics, although it is not clear that this
has caused a problem for many ontology developers or users. The
standard approach has been to use a set of binary relations as
specified in Table \ref{tab:characteristics-atemporal}, and to assume
an \OBOREL\ type interpretation for time.

\subsection{BFO2 and temporalized relations}

As part of the release process for the OWL translation of version 2 of
the Basic Formal Ontology (BFO)\cite{Grenon2004}, a number of people
explored different strategies for unifying OWL binary properties with
the ternary relations in the BFO2 reference
specification\cite{Grewe}. The goal was to find a way of representing
the temporal aspects of BFO2 relations in OWL in a formally satisfying
way.  One such strategy is the \emph{temporalized relations} (TR)
strategy, in which each reference relation relating continuants has
two or more OWL cognates, \pr{rel-at-some-times} and
\pr{rel-at-all-times}. This strategy has been adopted as the official
one for the OWL translations of BFO2\cite{Graz}.

\subsection{Outline}

In this review I do not attempt to compare or even describe the
different modeling possibilities; instead I focus purely on the
temporalized relations strategy, due to pressure to make this the
standard for OBO library ontologies. I first provide an outline of
temporalized relations, drawing on the existing release notes and
documentation, attempting to fill some gaps and provide additional
explanations of some of the complexities. I then present the major
problems posed by these relations:

\begin{enumerate}

\item Temporalized relations fail to capture the biological reality, forcing
  ontology editors to make a choice between two unsatisfactory models.

\item Within the model there is no facility for modeling important classes that lack
  the 'rigidity' criterion.

\item Temporalized relations are confusing for both users and
  experienced ontology editors.
\end{enumerate}

These problems are inter-related; even experienced ontology editors
may not understand the choices they are being asked to make when
migrating to temporalized relations.

Finally I end with some specific recommendations regarding the
temporalized relation strategy in general and BFO2 specifically.

My intentions are to make this review accessible to a wide audience,
and to keep logical formulas to a minimum. This is difficult because
one of the problems with the TR strategy is that it forces complexity
upon both the users and developers of an ontology, necessitating some
discussion of that complexity in order to explain why it does not
yield the intended benefits. I have annexed some of the finer grained
details into an appendix.

\begin{table}
\begin{tabular}{ | p{2.5cm} | p{1cm} | p{1cm} | p{2.5cm} | }
\hline
\textbf{Relation} & \textbf{Trans} & \textbf{Symm} & \textbf{Inverse Of}  \\
\hline
\partOf\  & Yes & No & \hasPart  \\
\hline
\hasPart\ & Yes & No & \partOf \\
\hline
\adjacentTo\ & No & Yes &  \\
\hline
\end{tabular}
\caption{Relation characteristics, atemporal. These are the characteristics of the instance level relations in the current RO (http://obo-relations.googlecode.com)}
\label{tab:characteristics-atemporal}
\end{table}


\section{Temporalized Relations}

Here I distinguish between reference relations (RRs) and their
manifestation in OWL as binary temporalized relations (TRs), using the
temporalized relation strategy (TRS). All relations are instance
level. As a typographic convention I use dashes to separate the words
in a temporalized relation, and underscores in a reference relation.

\subsection{Translation template}

The BFO2 Graz version release notes\cite{Graz} specify a general
template for relating RRs to TRs:

\begin{verbatim}
x rel-at-some-time y ->
 exists(t) exists_at(x,t) -> 
    exists_at(y,t) and rel(x,y,t)
x rel-at-all-times y ->
  forall(t)  exists_at(x,t) -> 
      exists_at(y,t) and rel(x,y,t)
\end{verbatim}

Here we focus on \partOf\ as an exemplar relation, whilst recognizing
that similar patterns may apply to other, but not all relations.

For \partOf\ connecting two continuants there are in fact \emph{three}
TRs rather than two (for reasons that will be explained shortly):

\begin{enumerate}

\item \partOfAtSomeTimes

\item \partOfAtAllTimes

\item \partOfAtAllTimesForWhichWholeExists

\end{enumerate}

Note that in BFO2 the actual labels are ``part of continuant at some
time'', ``part of continuant at all times'' and ``part of continuant
at all times for which whole exists'', we shorten this for brevity --
here we are only concerned with relations that involve a continuant.

\subsection{TRs force a different interpretation from OBOREL}


On the surface, the \atAllTimes\ form appears to be the same as the
\OBOREL\ interpretation. For example, the following OWL axiom may
appear in an ontology that uses TRs:

$$
\CellNucleus\ \pr{SubClassOf}\ \partOfAtAllTimes\ \pr{some}\ \Cell
$$

This may seem to be the same as the statement ``every cell nucleus is
part of some cell at all times''. We might even be able to
automatically translate an ontology written using the \OBOREL\
interpretation into TRs. \emph{However, these are NOT the same, and an
  understanding of why this is so is crucial if these relations are to
  be used correctly, and to understand the long-term consequences of
  using them}.

It is not the case that TRs are the same as what has come before, but
with longer labels. \emph{The semantics are fundamentally
  different}. Whereas \OBOREL\ allowed permanent generic parthood (in
which a nucleus must always be part of a cell, but can be transferred
between cells), with TRs that possibility is disallowed. See Table
\ref{tab:fol-class-axioms} in the appendix for details. I will return
to this in the evaluation.

The fact that these are different is of utmost importance to how
ontologies are created, and affects the \emph{characteristics} of
these relations in some ways that might seem surprising.

\subsection{Object property characteristics}

In OWL, relations (object properties) can have certain characteristics
such as being \emph{transitive}, \emph{symmetrical}, and they may be
related to other relations via logical axioms such as \emph{inverse
  properties} and property chains. These are extremely useful for many
purposes - transitivity has been at the core of bioinformatics
applications of ontologies from the initial version of the Gene
Ontology\cite{Ashburner2000}, if not before. Inverse properties are
useful for instance level reasoning, and for finding errors in complex
ontologies.

When translating a RR to a TR, it may not be immediately clear what
properties of the RR should be carried over to the TR. If a reference
relation $R$ has an inverse $R'$, it doesn't follow that the
temporalized versions will be inverses. Ideally we would be able to
prove that the RR and TR are consistent, although the proof may be
obvious to an expert logician. Here I use \partOf\ as an illustrative
example, although each relation may require individual examination for
its properties.

Table \ref{tab:characteristics-temporalized} shows some of the
property characteristics of the various forms of the continuant
parthood relations. Figure \ref{fig:part} shows this in graphical
form.

%----------------------------------------
\begin{figure}
\center
\includegraphics[width=7cm]{part}
\caption{Parthood relations, both atemporal (A) and temporalized (B)}
\label{fig:part}
\end{figure}
%----------------------------------------


\begin{table}
\begin{tabular}{ | p{3cm} | p{1cm} | p{4cm} | }
\hline
\textbf{Relation} & \textbf{Trans} & \textbf{Inverse Of}  \\
\hline
\partOf\ \atSomeTimes & No & \hasPart\ \atSomeTimes \\
\hline
\partOf\ \atAllTimes & Yes & \hasPartAtAllTimesForWhichPartExists \\
\hline
\partOfAtAllTimesForWhichWholeExists & Yes & \hasPart\ \atAllTimes \\
\hline
\hasPart\ \atSomeTimes & No & \partOf\ \atSomeTimes \\
\hline
\hasPart\ \atAllTimes & Yes & \partOfAtAllTimesForWhichWholeExists \\
\hline
\hasPartAtAllTimesForWhichPartExists & Yes & \partOf\ \atAllTimes \\
\hline
\end{tabular}
\caption{Relation characteristics for core continuant parthood relations. These characteristics are declared in the current BFO2 OWL Graz version}
\label{tab:characteristics-temporalized}
\end{table}

\subsubsection{Transitivity}

In the case of the RR \partOf\ (which is transitive), the stronger
\atAllTimes\ TR retains the transitivity characteristic, whereas the
weaker \atSomeTimes\ TR does not have this. This means the weaker
version is often safer to use in an ontology, but will lead to fewer
inferences.

\subsubsection{Symmetricality}

For other relations and other characteristics, the translation may not
be obvious. For example, the adjacency relation is commonly assumed to
be symmetric\footnote{we are only considering instance level
  relations}. Should this symmetricality characteristic be carried
over to the temporalized form?

Currently BFO2 does not have an adjacency relation, or any other
reference relation that is symmetric, so the following is based on my
own understanding, and is shown in \ref{tab:characteristics-adj}. I would assume that the symmetricality should be
declared for the \atSomeTimes\ form and not the \atAllTimes\ form. In
contrast to the \partOf\ and \emph{transitivity}, here it is the
\emph{weaker} form of relation that inherits the characteristic.

\subsubsection{Sub Properties}

We assume a hierarchy in which \atSomeTimes\ is the most general, with
\atAllTimesForWhichSubjectExists\ intermediate, and \atAllTimes\ most
specific. This is illustrated for parthood in figure \ref{fig:part}.

Note how this interacts with other properties. If:
$$x\ \adjacentTo\ \atAllTimes\ y$$

then we can infer that 

$$y\ \adjacentTo\ \atSomeTimes\ x$$.

\subsubsection{Inverse Properties}

Ontologies frequently declare inverse relations. For example, the
RR \partOf\ is the inverse of \hasPart. This is fairly standard
practice, and the inverse relations are extremely useful for
reasoning. In OWL, it's not strictly necessary to declare an inverse,
as it is possible to use an InversePropertyExpression. Here we take
the view that inverses that have typically been declared in previous
ontologies are useful, and should also be declared as RRs and have
corresponding TRs.

The inverse of the TRs of \partOf\ may not be completely
intuitive. Naively we might guess that \partOfAtAllTimes\ would be the
inverse of \hasPartAtAllTimes, but this not the case.

In fact, the declaring the inverse of \partOfAtAllTimes\ requires
declaring a third TR form, \hasPartAtAllTimesForWhichPartExists. The
\atAllTimesForWhichSubjectExists\ form can be generated for some but
not all RRs. Whilst it complicates the ontology of TRs to introduce
this extra form, the alternative of not having inverses (or of being
forced to write complicated inverse expressions) may be too
prohibitive for many users.




\subsubsection{Other characteristics}

We do not consider other characteristics such as anti-symmetry and
domain/range restrictions here.


\subsubsection{Generating a TR from an RR}

Currently there is no ``recipe'' for generating a set of TRs from an
RR. Different patterns may apply to different RRs.

For example, when creating the TRs for inheresIn and its inverse
bearerOf, it is correct to declare bearerOfAtAllTimes InverseOf
inheresInAtAllTimes.




\begin{table}
\begin{tabular}{ | p{3cm} | p{1cm} | p{4cm} | }
\hline
\textbf{Relation} & \textbf{Symm} & \textbf{Inverse Of}  \\
\hline
\adjacentTo\ \atSomeTimes & Yes & \adjacentTo\ \atSomeTimes \\
\hline
\adjacentTo\ \atAllTimes & No & \\
\hline
\end{tabular}
\caption{Relation characteristics for a typical symmetric relation. This relation is not part of the BFO2 OWL Graz version}
\label{tab:characteristics-adj}
\end{table}


\section{Evaluation}

\subsection{Temporalized Relations do not reflect the intentions of
  ontology editors or the underlying biology}

TRs present many challenging problems. For example, when converting an
anatomy ontology that has been modeled traditionally using the
relations in table \ref{tab:characteristics-atemporal}, the ontology
editor must make a choice on a case by case basis as to which of
the relations in \ref{tab:characteristics-temporalized} should be
used.

This is an onerous task, but this could be justified if the results
were better ontologies. However, in many cases \emph{none of the
  choices are appropriate}.

This is because for many ontologies, the most appropriate choice of
parthood relationship is the \emph{permanent-generic} form, as
specified in the original OBO relations paper. The standard example
here is the relationship between a cell nucleus and a cell. At any
moment in time, a given cell nucleus is by definition part of some
cell\footnote{we would consider extruded nuclei to be transformations
  of cell nuclei, but instantiating a different class}. However,
\emph{this need not be the same cell throughout the lifetime of the
  nucleus}.

In contrast, if the \partOfAtAllTimes\ TR is used then the
interpretation is that the cell nucleus is always part of the same
cell. This interpretation can be proved to be formally wrong in cells
that undergo cell division\cite{CellDiv}. This is because in reality
the cell nucleus is always part of a cell, \emph{but not the same
  cell}. With \partOfAtAllTimes\, there is no ``migration'' allowed -
a nucleus is always part of the same cell.

Understanding the consequences of incorrectly modeling the
relationship in this way are not straightforward. In some situations
there may be no severe problems, if the ontology is not used for
instance data. However, it would be unusual to adopt a far more
complex formalism (TRs) in the name of formality only to arrive at an
ontology that is formally incorrect.

Given that the ontology editor can not use \partOfAtAllTimes\ without
making a false statement, they must choose a weaker qualified version
of \partOf\ such as \partOfAtSomeTimes. Such a usage would be formally
valid, but incomplete from the point of view of useful reasoning. This
is because the weaker \atSomeTimes\ form lacks the transitivity
characteristic. I assume there to be no disagreement that this is a
completely unsatisfactory solution, as most ontologies are dependent
on parthood transitivity.

A third possibility is to use
the \partOfAtAllTimesForWhichWholeExists\ form, but this would also be
false [note the above proof needs to be extended to cover this case].

In the nucleus-cell case, the ontology editor requires the permanent
generic form in order to be both accurate and to get the required
inferences. \emph{However, this form is specifically excluded in the
  TR strategy}.

The cell nucleus example is the standard one, because it is central to
all of eukaryotic biology. It is by no means the only such
example. The problem arises whenever we have material passed around
from one carrier to another.  Enumerating a list of examples is
difficult because the instance level identity conditions may not be
clear. 

The problem is not limited to parthood relations. Use of TRs requires
that all continuant relations are temporalized. This includes
relations used to classify structures by phenotype.

\subsection{Rigidity requirement is too onerous}

The BFO2 Graz release notes state:

\begin{quotation}

  Thus we only instantiate ``rigid'' classes, as the interpretation we
  take is a rdf:type C => forall(t) a exists at t -> a instance of C
  at t. Temporally restricted instantiation is not supported in this
  version of BFO in OWL. We are working on it for the future.

\end{quotation}

To many users this may seem like an obscure point, but it is actually
quite a sever restriction. A class is rigid if it is instantiated
``for life''. If an individual transforms from being an instance of
one class of thing to another, then those classes are not rigid.

An example of a rigid class may be ``Homo sapiens''. If an individual
instantiates this class at some time t, then they instantiate it all
times for which they exist (barring some unusual inter-species
transformation).

This clause means material entity classes such as the following may
not be supported in this version of BFO:

\begin{enumerate}

  \item 'human with Parkinson's disease'
  \item 'female organism'
  \item 'infected lung'
  \item 'professor' (but \emph{professor role} is allowed)
  \item 'human patient' (but \emph{patient role} is allowed)
  \item 'oocyte'
  \item 'fractured bone'
  \item 'happy human'
  \item 'fetal heart'
  \item 'neural crest cell'
  \item 'open heart valve'
  \item 'gravid uterus'
  \item 'phosphorylated EGFR protein'
  \item 'cytoplasmic NFkB'
  \item any leaf node from PATO

\end{enumerate}

In some cases the constraint may not be so onerous. It can be argued
that a well structured ontology would never include a class
``professor'', and that this should always be modeled using a rigid
class (human) plus a role (professor role). However, these decisions
should be made on a case by case basis by each ontology that than
imposed from the upper ontology.

In other cases the distinction between rigid and non-rigid may not be
clear. One can argue that when an EGFR protein changes state from
being unphosphorylated to a phosphorylated state it is no longer the
same instance - the protein literally ceases to exist and is replaced
by a distinct individual an instant later, sharing all the same
properties except that it is phosphorylated. In fact one could take
this position for all of the above cases, in which case the TR
strategy becomes similar to the temporally qualified continuant (TQC)
strategy. I do not explore this further, as I assume this is contrary
to the expectations of the TR proponents.

When considering BFO2 in particular, there is an interesting
disjunction between the reference document, which explicitly states
that \emph{determinates} (for example, qualities such as ``square'',
``charged'', ``cylindrical'') are non-rigid. The Graz release states
that instantiation of these classes is not supported. These two
seemingly contradictory statements are not explicitly linked
anywhere. The modeling implications of this disconnect are not clear,
and require further documentation. It cannot be ruled out that this
restriction will involve further complexity.

It may be the case that future versions of the TR strategy will allow
for non-rigid classes. It is not clear how this will be achieved
without additional complexity.  The TR strategy must be evaluated on
what exists presently, and at this time the strategy comes with
constraints that ontology developers should be fully aware of.

\subsection{Temporalized relations add complexity}

The most striking feature of an ontology that uses the TR strategy is
the complexity. Whereas using traditional modeling, we may have has a
single parthood relation, we now have three. Both users and developers
have to have a strong understanding of these to use them correctly.

\subsubsection{Multiple levels of quantification}

Another source of complexity is that ontology editors now have to
handle an extra layer of quantification. Consider the possible ways to
model the relationship between a population of organisms and an
organism:

\begin{verbatim}
population SubClassOf 
  has-part-at-some-time some organism
population SubClassOf 
  has-part-at-all-times some organism
population SubClassOf 
  has-part-at-all-times-that-part-exists some organism
population SubClassOf 
  has-part-at-some-time only organism
\end{verbatim}

In each case there is in fact three levels of quantification. The
first level is the OWL subclass axiom, which states that the condition
holds for ALL instances of a population. Also within the scope of OWL
is the final SOME or ONLY quantifier. Finally, embedded within the
relation is an additional layer than is opaque to OWL reasoners (and
thus harder to use standard tools to check), i.e. the temporal
qualification.

This is in contrast to the simpler, well-documented kind of atemporal
quantification ontology developers perform at the moment.

I have a great deal of experience in training and assisting ontology
developers in the use of tools such as reasoners and in making the
transition to OWL. In my estimation, the level of complexity TRs exert
is simply too much.

\subsubsection{Migration is complex and will be error-prone}

In theory some of this complexity could be tamed by additional tooling
(although it is not clear who has the resources to implement
this). However, even if this complexity can be hidden from the user,
the ontology developer is forced to wrestle with the complexity.

This complexity first manifests when an ontology developer chooses to
migrate from a traditionally modeled ontology using relations from
table \ref{tab:characteristics-atemporal}, assuming an \OBOREL\
interpretation.

For each axiom that uses a relation that has multiple variants in TR
form, the developer must make a choice of which of the 3 variants to
use. Currently there is no documentation or guidance on how they
should do this.

A conservative strategy would be to convert all continuant relations
to the \atSomeTimes\ form. This would result in the ontology being far
less useful for inference (due to the loss of properties such as
transitivity). However, we might expect it to be at least valid, since
the relation is weaker.

In fact this is not universally safe. Consider the OWL axiom:

\begin{verbatim}
(\partOf\ \some\ \pr{nucleus}) \pr{DisjointWith} (\partOf\ \some\ \pr{cytoplasm})
\end{verbatim}

Converting these to \atSomeTimes\ would actually result in an axiom
that is \emph{too strong}. Conversely, converting to \atAllTimes\
would be too \emph{weak}, because it would admit the possibility of
migratory structures being part of two spatially disconnected
locations at the same time (so long as they weren't permanently part
of each). It is not clear how the ontology maintainer should convert
this axiom -- \emph{because all choices are suboptimal}.

Even in cases where there is an optimal way to translate to TRs,
performing the conversion requires an ontology developer who has a
strong understanding of the domain and of the logic.

\subsection{Case study: HDOT}

...

\section{Discussion}

\subsection{Bad smells}

Software engineers have a concept of a ``bad smell'' in code. This is
often applied when a piece of software or library seems overly baroque
or complex. Whilst this sense may be at least in part subjective, many
software engineers work on this intuitive level, and will typically
avoid using a piece of software if they perceive it to have a ``bad
smell''. Sometimes there is external pressure to use the software in
question - this can sometimes prolong its life but in general software
engineering solutions perceived to have this quality are typically
replaced by something more efficient and simpler.

From various conversations with experienced ontology engineers,
bioinformaticians and logicians, I feel confident in saying that the
TR solution as embodied in BFO2 is generally perceived to have a ``bad
smell'' in the same way. I include this as a discussion point rather
than part of a formal evaluation since I can't produce hard evidence
of this quality. However, proponents of the TR solution ignore this
engineers' intuition at their peril: it hinders the adoption of the
relations as standards, and even if this is forced from above,
developers will work around them. I know of various groups with plans
to simply rewrite TRs back into their standard form should they become
adopted.

\subsection{Recommendations}

\subsubsection{Do not use} My primary recommendation is that
Temporalized Relations should not be used as a replacement for
existing atemporal relations. Ontologies should in general not migrate
to them.

\subsubsection{Documentation} The TR strategy needs much more
documentation if ontology developers are to use TRs. Even if TRs are
abandoned in their current form (as I recommend), more documentation
would be useful to be able to help achieve consensus on this matter.

\subsubsection{Alternate strategies} Given the inherent limitations
and complexity of TRs, adequate consideration should be given to
alternate strategies such as Temporally Qualified Continuants
(TQCs). The ``default'' strategy of continuing to use simple OWL
object properties as if they has a \OBOREL\ interpretation should be
the default strategy until an adequate replacement is found.

\subsubsection{Use cases} If adopted, TRs will require tremendous
effort in ontology migration, documentation and tooling. There is
little to motivate ontology developers to do this as the existing
default strategy works for them. The main motivating factor seems to
be a desire for formal correctness, at the expense of usability and
biological correctness. 

\subsubsection{Road map} The existing TR proposal embodied in the BFO2
Graz release is not complete. For example, of non-rigid classes, the
release notes state that ``we are working on [non-rigid classes] for
the future''. There shoould be a roadmap indicating when these
solutions are expected to transpire. Ontology developers should not be
expected to commit production ontologies to an experimental project
with no roadmap.

\subsubsection{Smooth transition} If TRs are to be adopted, there
needs to be some incremental transition plan for migration.

\section{Conclusions}

Temporalized Relations would be a fundamental change to the way
relationships are modeled in ontologies. They would introduce
significant additional complexity to both users and developers of
ontologies.

Some of these costs could be justified if Temporalized Relations were
on a path towards making ontologies more biologically
accurate. However, there are no mativating use cases for this
transition, and in fact migrating to Temporalized Relations would lead
to ontologies becoming \emph{less} accurate, in addition to more
confusing. My recommendation is unambiguous in its rejection of the
use of Temporalized Relations in biological ontologies.

\section*{Acknowledgments}



%\section*{References}

% ========================================
\bibliography{trc}
\bibliographystyle{plain}
% ========================================

\newpage
\section*{Appendix}

\subsection{Non-rigid classes}

A class $c$ is non-rigid if there exists an instance $i$ that exists
at $t_1$ and exists at $t_2$, and it is not the case that $i$
instantiates $c$ at $t_1$, and $i$i instantiates $c$ at $t_2$. $t_1$
may precede or succeed $t_2$.

\begin{enumerate}

\item 'human with Parkinson's disease' -- a person is not born with
  Parkinsons (although they may be born with genes that
  predispose). It is possible for a human being $i$ who exists at
  $t_1$ and $t_2$ to not instantiate human-with-PD at $t_1$ and to
  instantiate human-with-PD at $t_2$. Therefore human-with-PD is
  non-rigid.

\item 'female organism' -- Some organisms (e.g. some species of
  arthropod) can change sex during their lifetime. It is possible for
  some such instance $i$ to instantiate 'male organism' at $t_1$ and
  then instantiate 'female organism' at $t_2$. If these classes are
  disjoint, then 'female organism' is non-rigid. Note that it is
  possible to define different sex concepts (gender, karyotypic sex,
  biological sex, ...), different arguments can be made about the
  rigidity of the corresponding material entity classes.

\item 'infected lung' -- it is possible for a lung $i$ to have the
  quality of being infected (alternatively: be the location of a
  population of invading organisms) at $t_1$, and then non-infected at
  $t_2$. Therefore the class 'infected lung' is non-rigid.

\item 'professor' -- It is possible for an individual $i$ to
  instantiate 'human with professor role' at one time, and then not
  instantiate this at some later time. Therefore 'professor' (as a
  material entity) is non-rigid. professorhood is best represented as
  a role that can be gained or lost.

  \item 'human patient' -- see 'professor'

  \item 'oocyte' -- 

  \item 'fractured bone' -- It is possible for some bone $i$ to
    instantiate the class 'non-fractured bone' at $t_1$ and then
    'fractured bone' at some later time $t_2$. Therefore 'fractured
    bone' (in the sense of a material entity - a bone that has the
    quality of being fractured) is a non-rigid class. An argument can
    be made that $i$ ceases to exist when it becomes fractured, and is
    replaced by a new individual $i_2$ at $t_2$. This could certainly
    be argued for severe breakages, where $i$ is replaced to two or
    more bone shards.

  \item 'happy human' -- It is possible for a human being $i$i to
    instantiate the class 'happy human' at $t_1$ (by virtue of bearing
    a happy disposition) and then 'unhappy human' at $t_2$. Therefore
    'happy human' is a non-rigid class.

  \item 'fetal heart' -- it is possible for a heart $i$ to instantiate
    'fetal heart' at $t_1$ and and then 'newborn heart' at
    $t_2$. Therefore 'fetal heart' is non-rigid. An argument can be
    made that $i$ is ceases to exist and is replaced by a new instance
    $i_2$ at $t_2$, but this would be unusual. An argument could also
    be made that there is no need for a class 'fetal heart' - the
    concept should be described using a rigid class 'heart' together
    with an occurrent 'fetal stage'. However, this would be a severely
    onerous penalty on many anatomy ontologies which frequently use
    stage as a differentia.

  \item 'neural crest cell' --

  \item 'open heart valve' -- it is possible for some heart valve $i$
    to instantiate 'open heart valve' at $t_1$ (by virtue of bearing
    the quality 'open' at this time, or, alternatively, by virtue of
    their being a lumen in the vessel) and then to instantiate 'closed
    heart valve' at $t_2$. Therefore 'open heart valve' is a non-rigid
    class.

  \item 'gravid uterus' -- it is possible for some uterus $i$ to
    instantiate a class 'non-gravid uterus' at time $t_1$ (by virtue
    of not being the location of a developing organism), and then
    instantiate a class 'gravid uterus' at some later time
    $t_2$. Therefore 'gravid uterus' is a non-rigid class.

  \item 'phosphorylated EGFR protein' -- there are different ways to
    model this depending on identity conditions on the instance level
    (we take identity conditions on the class level as being
    uncontroversial - class equivalence is determined by structure for
    molecules). Using model $M_1$, we assume there to be a single
    instance $i$ of an EGFR protein which transitions through
    different states. Here, $i$ instantiates 'unphosphorylated EGFR'
    at $t_1$ and then later the same instance $i$ instantiated
    'phosphorylated EGFR' at $t_2$. Under this model, 'phosphorylated
    EGFR' is non-rigid. We can model this differently - call this
    $M_2$. Here $i_1$ instantiates 'unphosphorylated EFGR' at
    $t_1$. Then, as a phosphate group is added at $t_2$, $i_1$ ceases
    to exist and its place is taken by $i_2$, which instantiates
    'phosphorylated EGFR'. Here $i_1$ and $i_2$ might be related via
    some relation such as 'transformation of'. This illustrates that
    any non-rigid class can be made rigid by changing instance-level
    identity conditions. At one extreme we can see life as a series of
    snapshots, with individuals living for an instant before being
    replaced by a doppleganger.

  \item 'cytoplasmic NFkB' -- this is similar to the EGFR case. Here
    the differentia is location. No structural change need take place.

  \item any leaf node from PATO -- Examples: square, open,
    cylindrical, hot, cold. This is a multi-faceted topic and a
    thorough discussion should wait until there is documentation on
    how to model quantities in BFO2.

\end{enumerate}

\subsection{First order logic axioms}

This section contains some additional material on the first order
logic axioms supporting the Temporalized Relations. 

SECTION NOT COMPLETE IGNORE FOR NOW

\begin{table}
\begin{tabular}{ | p{3cm} | p{4cm} | }
\hline
\textbf{Relation} & \textbf{Axiom}  \\
\hline
$$x\ \pr{part-of-}\ \atAllTimes\ y$$ &
        \tbleqn{
 & \dimp  \A t \existsAt(x,t) \imp\ \\
 & \existsAt(y,t), \partOf(x,y,t)
} \\
\hline
$$x\ \pr{part-of-}\ \atSomeTimes\ y$$ &
        \tbleqn{
 & \dimp  \E t \existsAt(x,t) \imp\ \\
 & \existsAt(y,t), \partOf(x,y,t)
} \\
\hline
\end{tabular}
\caption{Temporalized relations axioms for parthood relations. Taken from \cite{Graz} and transcribed into FOL syntax}
\label{tab:fol-part-temporalized}
\end{table}


\begin{table}
\begin{tabular}{ | p{1.8cm} | p{6.2cm} | }
\hline
\textbf{Axiom}  \\
\hline
\partOf (\OBOREL) &
        \tbleqn{
 & \A x \A t : \instanceOf(x, \CellNucleus, t) \imp \\
 & \E y : \instanceOf(y, \Cell, t), \partOf(x,y,t)
} \\
\hline
\pr{part-of-} \atAllTimes &
        \tbleqn{
 & \A x \E t : \instanceOf(x, \CellNucleus) \imp \\
 & \E y \instanceOf(y, \Cell), \A t \existsAt(x,t) \imp\ \\
 & \existsAt(y,t), \partOf(x,y,t)
} \\
\hline
\pr{part-of-} \atSomeTimes &
        \tbleqn{
 & \A x \E t : \instanceOf(x, \CellNucleus) \imp \\
 & \E y \instanceOf(y, \Cell), \E t \existsAt(x,t) \imp\ \\
 & \existsAt(y,t), \partOf(x,y,t)
} \\
\hline
\end{tabular}
\caption{
  Semantics of class axioms with parthood example. The first row shows the
  biologically correct relationship (permanent generic parthood), given by \OBOREL\ semantics. The next two rows show 
  two of the temporalized options - neither of these is equivalent to the \OBOREL\ version. 
}
\label{tab:fol-class-axioms}
\end{table}


\end{document}
